\documentclass[a4,12pt,dvipdfmx]{jreport}

\usepackage{wadakensoturon}
\usepackage[dvipdfmx]{graphicx}
\usepackage{amsmath}
\usepackage{array,booktabs}
\usepackage{subfig}
\usepackage{multirow}
\usepackage{comment}
\usepackage{subfig}
\usepackage{here}
\usepackage{lscape}
\usepackage{cite}
\usepackage{url}
\usepackage{cases}
\usepackage{empheq}
\usepackage{float}
\usepackage{multirow}
%\usepackage{subfigure}
\allowdisplaybreaks

\setcounter{topnumber}{100}
\setcounter{bottomnumber}{100}
\setcounter{totalnumber}{100}
\renewcommand{\textfraction}{0.0}
\renewcommand{\topfraction}{1.0}

\numberwithin{equation}{chapter}

\usepackage{listings} %日本語のコメントアウトをする場合jlistingが必要
%ここからソースコードの表示に関する設定
\lstset{
  basicstyle={\ttfamily},
  identifierstyle={\small},
  commentstyle={\smallitshape},
  keywordstyle={\small\bfseries},
  ndkeywordstyle={\small},
  stringstyle={\small\ttfamily},
  frame={tb},
  breaklines=true,
  columns=[l]{fullflexible},
  numbers=left,
  xrightmargin=0zw,
  xleftmargin=3zw,
  numberstyle={\scriptsize},
  stepnumber=1,
  numbersep=1zw,
  lineskip=-0.5ex
}
%ここまでソースコードの表示に関する設定


%著者の情報
\title{\mbox{カレントミラーを組み合わせた}\mbox{折り返し型アナログ乗算回路の}\mbox{出力範囲を拡大する回路構成}}
\month{2024年2月}
\year{2024}
\date{2024年2月日}
\id{1512201217}
\author{小島\hspace{1zw}光}
\supervisor{和田\hspace{1zw}和千}
\labolatory{波動信号処理回路研究室}

\company{就職先の会社名}
\companyaddress{会社の所在地}
\companypostcode{会社の郵便番号}
\companyTEL{会社の電話番号}

\address{}
\postcode{}
\TEL{}

%\bkm%				PDFにブックマークをつけるには,ここをコメントアウトしてください.
%\com%			進学の方は,ここをコメントアウトしてください.
\titlecover%		表紙を作らない場合,ここをコメントアウトしてください.
%\chaptercover%		章の表紙を作らない場合は,ここをコメントアウトしてください.
%\sectionclearpage%	節の終わりで改ページしない場合は,ここをコメントアウトしてください.

\begin{document}
\maketitle[%表紙と目次を作成
%ここに図目次などの追加項目を書くことができます.
]

\chapter{序論}
現在、機械学習の分野ではディープラーニングが盛んに研究され、とくに時系列データを扱うためには各ノードが次の層への結合のみを持つ順伝搬型(feed-forword)ではなく以前の層への結合も持つリカレントニューラルネット(RNN)と呼ばれる方式が使われている。しかし、一般のディープラーニングにおいて、現実的な精度を得るためには多くの隠れ層を必要とするがその学習は必ずしも現実的な計算量では終わらない。これは多数の隠れ層を持つニューラルネットワークでは、ノード間の結合が非線形な活性化関数で定義されるため、教師データに対する数値解析的なフィッティングしかできないことに起因する。そこで、学習効率を上げる方法としてRNNを物理現象によって再現したリザバが登場した。\par
このリザバからの複数の出力の積和演算をすることで特徴量を抽出することができる。特に、学習するのはリザバではなく積和演算の重みのみであるため学習コストを低減させることができる。今回、光で入出力を行うフォトニックリザバにおいては高速な画像認識に用いることができる可能性が報告されている。しかしながら、現状では出力の光を光のまま積和演算することは難しいためフォトダイオードによって電気に変換し積和演算・並びに学習を行うことが検討されている。ここで、積和演算が画像認識速度のボトルネックにならないよう高速な処理が必要なため今回はアナログ信号の積和演算を行うギルバート乗算回路を使用することが本学で検討されていた。ギルバート乗算回路を用いる積和演算では図\ref{fig:1_config}に示すよう、出力信号に対してギルバート乗算回路で重みづけを行い、出力信号をまとめることで電流的に和をとる。即ち、各乗算器の出力振幅が積和演算回路の出力振幅の上限となるため、積和演算を行う信号の数が増えるとその分入力範囲が限られる。これにより信号対雑音比(S/N比)の劣化が懸念される。\par
本研究では、S/N比向上のため乗算回路単体の出力振幅を拡大する構成を提案する。本論の構成は以下のとおりである。まず2章で検討していたギルバート乗算回路構成を示し、小信号解析を行うことでその特性を確認する。さらにその現実的な出力範囲を導出する。3章において提案する回路構成についての基本的な方針を示し、小信号解析において提案回路がギルバート乗算回路同様、アナログ信号の乗算が可能であることを明らかにする。また、その出力範囲を導出しギルバート乗算回路に対し出力範囲を広げられることを示し、シミュレーション上でも確認する。最後に、4章で結論並びに今後の展望について述べる。
\chapter{ギルバート乗算回路}
    \section{回路構成}
        まず、図\ref{fig:2_gilbert}に現在広く使用されているギルバート乗算回路の回路図を示す。
        \begin{figure}[!b]
            \begin{center}
                \includegraphics[width=0.99\textwidth]{figures/chapter2/gilbert.pdf}
                \caption{ギルバート乗算回路}
                \label{fig:2_gilbert}
            \end{center}
        \end{figure}
        ギルバート乗算回路は$v_{in}$と$V_{CTRL}$に差動で信号を入力することで二つの入力信号に比例した電圧出力を差動で出力することができる。ここで、$M_{B}$の差動対は入力信号に比例した電流を$M_{A}$のソースから引き込む。さらに$M_{A}$の動作点が制御電圧$V_{CTRL}$に比例して変動する。これにより負荷抵抗に流れる電流が二つの入力に比例していることが分かる。詳細は次節にて導出する。


    \section{小信号解析}
        \subsection{動作点の変動}   \label{ch:gilbert_valiable_gm}
            小信号解析を行うにあたり$M_{A}$の扱いが問題となる。小信号解析ではMOSFETの出力が線形に近似できるような小信号入力に対して行うが、動作点が変動すると線形な近似が合わなくなる。この問題を扱うために、まず図\ref{fig:2_OP}に示す差動ゲート接地の差動対について考える。\par
            \begin{figure}[!b]
                \begin{center}
                    \includegraphics[width=0.7\textwidth]{figures/chapter2/OperatingPoint.pdf}
                    \caption{ゲート接地の差動対}
                    \label{fig:2_OP}
                \end{center}
            \end{figure}
            \clearpage
            一般に、MOSFETのドレイン電流$I_{D}$は二乗則に従い、ゲートソース間電圧$V_{GS}$、しきい電圧$V_{th}$と形状などによって決まるトランスコンダクタンス係数$K$を用いて
            \begin{align}
                I_{D}=K(V_{GS}-V_{th})^{2}  \label{eq:2_id}
            \end{align}
            と表せる。さらに、MOSFETのトランスコンダクタンスはドレイン電流をゲートソース間電圧で偏微分したものであるので差動成分が$0$、つまり$V_{diff}=0$のときトランスコンダクタンスを$g_{m}$とすると式(\ref{eq:2_id})を用いて
            \begin{align}
                g_{m}&=\frac{ \partial I_{D} }{ \partial V_{GS} }   \notag\\
                &=2K(V_{GS}-V_{th})     \label{eq:2_gm}
            \end{align}
            となる。次に差動成分$V_{diff}\neq0$のとき、左右のMOSFETのトランスコンダクタンス$g_{ml},g_{mr}$は
            \begin{align}
                g_{ml}&=2K\left\{ (V_{G}+V_{diff}) -V_{th} \right\}     \notag\\
                &=g_{m}+2KV_{diff}      \notag\\
                g_{mr}&=2K\left\{ (V_{G}-V_{diff}) -V_{th} \right\}     \notag\\
                &=g_{m}-2KV_{diff}      \notag
            \end{align}
            と計算できるので、$2KV_{diff}\equiv\Delta g_{m}$とおけば
            \begin{align}
                g_{ml}&=g_{m}+\Delta g_{m}   \label{eq:2_dgml}\\
                g_{mr}&=g_{m}-\Delta g_{m}   \label{eq:2_dgmr}
            \end{align}
            と表すことができる。$\Delta g_{m} \propto V_{diff}$であることから、図\ref{fig:2_OP}のようなゲート接地差動増幅回路においてトランスコンダクタンスはゲートソース間電圧の差動成分に比例することが分かった。
            \newpage
            
        \subsection{小信号等価回路}
            次に小信号等価回路を考えるが、ここでギルバート乗算回路は差動対の組み合わせでできているため、半回路を考えることで回路全体の小信号解析を行うことができる。半回路では対応する電位・電流はそれぞれ逆符号となるため半分のみを考えれば全体を解析したことと等価になる。そこで左右の負荷抵抗$R_{L}$、$\mathrm{M_{A1}}$と$\mathrm{M_{A4}}$、$\mathrm{M_{A2}}$と$\mathrm{M_{A3}}$、$\mathrm{M_{B1}}$と$\mathrm{M_{B2}}$がそれぞれ対応すしていることに留意すると図\ref{fig:2_gilbert_half}のような部分において半回路を考えれば良いと分かる。
            \begin{figure}[!b]
                \centering
                \includegraphics[width=0.99\textwidth]{figures/chapter2/gilbert_half.pdf}
                \caption{ギルバート乗算回路の内半回路を考える部分}
                \label{fig:2_gilbert_half}
            \end{figure}
            図\ref{fig:2_gilbert_half}の半回路を小信号等価回路に置き換える際、MOSFETは図\ref{fig:2_moseq}のような小信号等価回路モデルに、直流電圧源は短絡に、直流電流源は開放に置き換えると図\ref{fig:2_half}のような小信号等価半回路を書くことができる。
            \begin{figure}[!b]
                \centering
                \includegraphics[width=0.5\textwidth]{figures/chapter2/mos_eq.pdf}
                \caption{MOSFETの小信号等価回路モデル}
                \label{fig:2_moseq}
            \end{figure}
            \begin{figure}[!b]
                \centering
                \includegraphics[width=0.99\textwidth]{figures/chapter2/halfeq.pdf}
                \caption{ギルバート乗算回路の小信号半回路}
                \label{fig:2_half}
            \end{figure}
            \clearpage
            ただし、$V_{S}$は差動のソース電位であり、一定であると捉えることができる。即ち$\mathrm{M_{C}}$は交流的な信号は$0\mathrm{ V}$であるため、接地として扱う。また$g_{mA},g_{mB}$は$\mathrm{M_{A},M_{B}}$のトランスコンダクタンス、$g_{dA},g_{dB}$は$\mathrm{M_{A},M_{B}}$のドレインソース間抵抗、$R_{L}$は負荷抵抗である。まず、電流$i_{1},i_{2}$はそれぞれ
            \begin{align}
                i_{1}&=(g_{mA}+\Delta g_{m})(-v_{AB})+g_{dA}(v_{outm}-v_{AB})        \label{eq:2_i1}\\
                i_{2}&=(g_{mA}-\Delta g_{m})(-v_{AB})+g_{dA}(v_{outp}-v_{AB})        \label{eq:2_i2}
            \end{align}
            と表せる。ここで、小信号等価回路に置き換えた際、$V_{A},V_{CTRL}$は短絡されているので、接点ABにKCLを用いると
            \begin{align}
                g_{mB}v_{in}+g_{dB}v_{AB}&=i_{1}+i_{2}    \notag\\
                &=-2g_{mA}v_{AB}-2g_{dA}v_{AB}               \notag\\
                v_{AB}&=-\frac{g_{mB}}{2g_{mA}+2g_{dA}+g_{dB}}v_{in}     \label{eq:2_vab}
            \end{align}
            と書ける。さらに、完全差動回路であることを踏まえると$i_{3}=-i_{2}$、$v_{outp}=-v_{outm}$という関係が成り立つ。このとき負荷抵抗$R_{L}$に流れる電流$i_{outm}$は
            \begin{align}
                i_{outm}&=i_{1}+i_{3}       \notag\\
                &=i_{1}-i_{2}   \notag\\
                &=-2\Delta g_{m}+2g_{dA}v_{outm}    \label{eq:2_ioutm}
            \end{align}
            と表せる。オームの法則を用いると
            \begin{align}
                v_{outm}&=-R_{L}i_{outm}    \notag\\
                &=\frac{ 2R_{L} }{1+2R_{L}g_{dA}}\Delta g_{m}v_{AB}     \notag
            \end{align}
            と計算できる。最終的な出力は$v_{outp}$と$v_{outm}$の差をとることとすると
            \begin{align*}
                v_{out}&:=v_{outp}-v_{outm}     \\
                &=-2v_{outm}                    \\
                &=-\frac{ 4R_{L} }{1+2R_{L}g_{dA}}\Delta g_{m}v_{AB}
            \end{align*}
            である。ここで、$g_{mA}>>g_{dA},g_{dB}$を仮定し\ref{ch:gilbert_valiable_gm}の結果と式(\ref{eq:2_vab})を用いると出力電圧は
            \begin{align}
                v_{out}=\frac{ g_{mB} }{ g_{mA} }\cdot \frac{ 4KR_{L} }{ 1+2R_{L}g_{dA} }\cdot V_{CTRL}\cdot v_{in}       \label{eq:2_vout}
            \end{align}
            と表すことができる。ここで、$V_{CTRL}$と$K$はそれぞれ$M_{A}$に与える制御電圧とトランスコンダクタンス係数である。以上より、小信号を入力した際には出力として入力電圧$v_{in}$と制御電圧$V_{CTRL}$に比例した電圧を得る、すなわち乗算ができることが確かめられた。
            \newpage


    \section{出力範囲}      \label{ch:2_range}
        次にギルバート乗算回路の出力範囲を考える。適切に乗算が行える条件はMOSFETが遮断領域に入らないことであるとすると、制約条件として各MOSFETにおいてしきい電圧$V_{th}$が一定であるとすると
        \begin{subequations}
            \begin{empheq}[left={\empheqlbrace}]{align}
                &V_{GS}-V_{th}<V_{DS}          \\
                &V_{th}<V_{GS}              
            \end{empheq}        \label{eq:2_binding_conditions}
        \end{subequations}
        を満たす必要がある。ただし、$V_{GS}$、$V_{DS}$、$V_{th}$はゲートソース間電圧、ドレインソース間電圧、しきい電圧である。これを図\ref{fig:2_gilbert}の各MOSFETに用いると
        \begin{subequations}
            \begin{empheq}[left={M_{A}:\empheqlbrace}]{align}
                &V_{A}+V_{CTRL}-V_{AB}-V_{th}<V_{out}-\frac{1}{2}v_{out}-V_{AB}           \\
                &V_{th}<V_{A}-V_{CTRL}-V_{AB}                               
            \end{empheq}        \label{eq:2_ma_binding_pre}
        \end{subequations}
        \begin{subequations}
            \begin{empheq}[left={M_{B}:\empheqlbrace}]{align}
                &V_{B}+v_{in}-V_{S}-V_{th}<V_{AB}-V_{S}      \\
                &V_{th}<V_{B}-v_{in}-V_{S}                  
            \end{empheq}        \label{eq:2_mb_binding_pre}
        \end{subequations}
        \begin{subequations}
            \begin{empheq}[left={M_{C}:\empheqlbrace}]{align}
                &V_{C}-V_{th}<V_{S}      \\
                &V_{th}<V_{C}           
            \end{empheq}        \label{eq:2_mc_binding_pre}
        \end{subequations}
        と表現することができる。$M_{A},M_{B}$の.a不等式には両辺にソース電位が含まれているのでそれを消去すると
        \begin{subequations}
            \begin{empheq}[left={M_{A}:\empheqlbrace}]{align}
                &V_{A}+V_{CTRL}-V_{th}<V_{out}-\frac{1}{2}v_{out}           \\
                &V_{th}<V_{A}-V_{CTRL}-V_{AB}                 
            \end{empheq}        \label{eq:2_ma_binding}
        \end{subequations}
        \begin{subequations}
            \begin{empheq}[left={M_{B}:\empheqlbrace}]{align}
                &V_{B}+v_{in}-V_{th}<V_{AB}      \\
                &V_{th}<V_{B}-v_{in}-V_{S}      
            \end{empheq}        \label{eq:2_mb_binding}
        \end{subequations}
        とまとめられる。さらに式(\ref{eq:2_mc_binding_pre})より
        \begin{align}
            0<V_{S}     \label{eq:2_range_vs}
        \end{align}
        である。次に式(\ref{eq:2_ma_binding})から$V_{A}$を消去すると
        \begin{align}
            V_{AB}+2V_{CTRL}<V_{out}-\frac{1}{2}v_{out}     \label{eq:2_range_va}\\
        \end{align}
        であり、また式(\ref{eq:2_mb_binding})から$V_{B}$を消去すると
        \begin{align}
            V_{S}+2v_{in}<V_{AB}        \label{eq:2_range_vb}
        \end{align}
        と分かる。以上より、式(\ref{eq:2_range_va}),(\ref{eq:2_range_va})から
        \begin{align}
            V_{S}+2v_{in}+2V_{CTRL}<V_{out}-\frac{1}{2}v_{out}      \label{eq:2_range_out}
        \end{align}
        である。さらに、出力電圧は電源電圧からも制約を受けるので
        \begin{align}
            V_{out}+\frac{1}{2}v_{out}<V_{dd}
        \end{align}
        であり、出力範囲を$v_{range}$とすると式(\ref{eq:2_range_out})と合わせ
        \begin{align}
            0<v_{range}<V_{dd}-(V_{S}+2V_{CTRL}+2v_{in})    \label{eq:2_girlbert_range}
        \end{align}
        と考えることができる。



%    \begin{figure}
%        \begin{center}
%            \includegraphics[width=125mm]{figures/chapter2}
%            \caption{}
%            \label{fig:2_}
%        \end{center}
%    \end{figure}

%    \begin{subequations}
%        \begin{empheq}[left={\empheqlbrace}]{align}
%        \end{empheq}        \label{eq:}
%    \end{subequations}

\chapter{カレントミラーを組み合わせた折り返し型アナログ乗算回路}

    \section{回路構成}
        序論で述べたS/N比向上のための方針として、折り返しカスコードの構成をとることが考えられる。図\ref{fig:3_folded_gilbert}に折り返しカスコード型の乗算回路を示す。
        \begin{figure}[!b]
            \begin{center}
                \includegraphics[width=0.99\linewidth]{figures/chapter3/folded_gilbert.pdf}
                \caption{折り返しカスコード型乗算回路}
                \label{fig:3_folded_gilbert}
            \end{center}
        \end{figure}
        この構造では定電流源を用いることで各$\mathrm{M_{B}}$に流れる信号電流の増減が$\mathrm{M_{p}}$に反転して伝えられる。ゲート接地増幅回路としてはpMOSFET($\mathrm{M_{p1}}$~$\mathrm{M_{p4}}$)を用いることで制御電圧に比例したトランスコンダクタンスを得る。このような折り返しカスコード型の構造をとることで出力電圧は$M_{p}$と電流源のpMOSFETで分圧するため、入力電圧分出力振幅が広がることが予測される。しかし左右それぞれにバイアス電流を流すため消費電流が増加する。さらにpMOSFETを小信号に使用するためどうしてもnMOSFETのみのギルバート乗算回路よりも遮断周波数が低下するといったデメリットが考えられる。$\mathrm{RHOM\;0.18\mu\;m\;Process}$にてギルバート乗算回路と折り返しカスコード型の乗算回路をバイアス電流が同じになるように設計したときのそれぞれの素子値を表\ref{table:3_gilbert_param}、表\ref{table:3_folded_gilbert_param}に示す。またこの素子値におけるシミュレーションでの周波数特性を図\ref{fig:3_gilbert_ac}、図\ref{fig:3_folded_gilbert_ac}にそれぞれ示す。
        \begin{table}[!b]
            \begin{minipage}[t]{.45\textwidth}
                \begin{center}
                    \caption{比較用に設計したギルバート乗算回路}
                    \label{table:3_gilbert_param}
                    \begin{tabular}{c|c|r}
                        \hline
                        \multicolumn{2}{c}{Gilbert}   & \multicolumn{1}{c}{Value}     \\
                        \hline\hline
                        &   Channel Length   &   0.72 $\mathrm{\mu m}$   \\
                        \cline{2-3}
                        $\mathrm{M_{A}}$   &   Channel Width   &   4.27 $\mathrm{\mu m}$   \\
                        \cline{2-3}
                            &   Multifinger   & 10    \\
                        \hline
                        &   Channel Length   &   0.72 $\mathrm{\mu m}$   \\
                        \cline{2-3}
                        $\mathrm{M_{B}}$   &   Channel Width   &   4.27 $\mathrm{\mu m}$   \\
                        \cline{2-3}
                            &   Multifinger   & 20    \\
                        \hline
                        &   Channel Length   &   0.72 $\mathrm{\mu m}$   \\
                        \cline{2-3}
                        $\mathrm{M_{C}}$   &   Channel Width   &   11.6 $\mathrm{\mu m}$   \\
                        \cline{2-3}
                            &   Multifinger   & 40    \\
                        \hline
                        \multicolumn{2}{c|}{$\mathrm{V_{dd}}$} &   1.8 $\mathrm{V}$   \\
                        \hline
                        \multicolumn{2}{c|}{$\mathrm{V_{A}}$} &   1.59 $\mathrm{V}$   \\
                        \hline
                        \multicolumn{2}{c|}{$\mathrm{V_{B}}$} &   1.09 $\mathrm{V}$   \\
                        \hline
                        \multicolumn{2}{c|}{$\mathrm{V_{C}}$} &   0.65 $\mathrm{V}$   \\
                        \hline
                        \multicolumn{2}{c|}{$\mathrm{R_{L}}$} &   300 $\mathrm{\Omega}$   \\
                        \hline
                    \end{tabular}
                \end{center}                
            \end{minipage}
            %
            \hfill
            %
            \begin{minipage}[t]{.45\textwidth}
                \begin{center}
                    \caption{設計した折り返しカスコード型乗算回路}
                    \label{table:3_folded_gilbert_param}
                    \begin{tabular}{c|c|r}
                            \hline
                            \multicolumn{2}{c}{Folded Cascode}   & \multicolumn{1}{c}{Value}     \\
                            \hline\hline
                            &   Channel Length   &   0.72 $\mathrm{\mu m}$   \\
                            \cline{2-3}
                            $\mathrm{M_{p}}$   &   Channel Width   &   20 $\mathrm{\mu m}$   \\
                            \cline{2-3}
                                &   Multifinger   & 10    \\
                            \hline
                            &   Channel Length   &   0.72 $\mathrm{\mu m}$   \\
                            \cline{2-3}
                            $\mathrm{M_{B}}$   &   Channel Width   &   4.27 $\mathrm{\mu m}$   \\
                            \cline{2-3}
                                &   Multifinger   & 20    \\
                            \hline
                            &   Channel Length   &   0.72 $\mathrm{\mu m}$   \\
                            \cline{2-3}
                            $\mathrm{M_{C}}$   &   Channel Width   &   11.6 $\mathrm{\mu m}$   \\
                            \cline{2-3}
                                &   Multifinger   & 40    \\
                            \hline
                            \multicolumn{2}{c|}{$\mathrm{V_{dd}}$} &   1.8 $\mathrm{V}$   \\
                            \hline
                            \multicolumn{2}{c|}{$\mathrm{V_{A}}$} &   1.59 $\mathrm{V}$   \\
                            \hline
                            \multicolumn{2}{c|}{$\mathrm{V_{B}}$} &   1.09 $\mathrm{V}$   \\
                            \hline
                            \multicolumn{2}{c|}{$\mathrm{V_{C}}$} &   0.65 $\mathrm{V}$   \\
                            \hline
                            \multicolumn{2}{c|}{$\mathrm{R_{L}}$} &   300 $\mathrm{\Omega}$   \\
                            \hline
                    \end{tabular}
                \end{center}
            \end{minipage}
        \end{table}
        \begin{figure}[!b]
            \centering
            \includegraphics[width=0.99\textwidth]{figures/chapter3/previous_ac.pdf}
            \caption{ギルバート乗算回路の周波数特性}
            \label{fig:3_gilbert_ac}
        \end{figure}
        \begin{figure}[!b]
            \centering
            \includegraphics[width=0.99\textwidth]{figures/chapter3/folded_ac.pdf}
            \caption{折り返しカスコード型の周波数特性}
            \label{fig:3_folded_gilbert_ac}
        \end{figure}
        \clearpage
        今回の設計ではトランスコンダクタンスも揃っているので利得は同程度であるが、遮断周波数は1桁程度落ちてしまっている。本論文での目的はS/N比を向上させるために出力範囲を拡大することであるが、フォトニックリザバに用いることを想定すると周波数特性が構造的にギルバート乗算回路よりも悪化するのは避けたい。そこでpMOSFETを使用せずに信号の折り返しを行うことで出力範囲を拡大できるのではないかと考え、これを実現する回路を図\ref{fig:3_folded_mirror_gilbert}に示し、これをカレントミラーを組み合わせた折り返し型乗算回路とした。\par
        \begin{figure}[!b]
            \begin{center}
                \includegraphics[width=0.99\textwidth]{figures/chapter3/NtoNFolded.pdf}
                \caption{カレントミラーを組み合わせた折り返し型乗算回路}
                \label{fig:3_folded_mirror_gilbert}
            \end{center}
        \end{figure}
        $\mathrm{M_{U},M_{L}}$はともに電流源として用いており、定電流$I_{Lbias}$を入力のMOSFETである$\mathrm{M_{B}}$とカレントミラーの参照電流を流す$M_{MB}$で分流する。これにより、入力の差動対によって$v_{in}$に比例した信号電流を符号を逆転させ$M_{MB}$に流す。カレントミラーのコピー側である$\mathrm{M_{MA}}$には$\mathrm{M_{MB}}$と$\mathrm{M_{MA}}$の形状比と$v_{in}$に比例した電流を流すことができる。これにより制御電圧を印加する$\mathrm{M_{A}}$に流れるバイアス電流を変動させる。$\mathrm{M_{A}}$はゲート接地増幅回路であり、\ref{ch:gilbert_valiable_gm}節での議論からトランスコンダクタンスは$V_{CTRL}$に比例しており、負荷抵抗に流れる電流を$V_{in}$と$V_{CTRL}$に比例したものにすることができる。そして負荷抵抗によって電流を電圧に変換する。このようにしてカレントミラーと差動対で電流を分流することで折り返しカスコード型の様に信号を伝達することができるのではないかと予測し、図\ref{fig:3_folded_mirror_gilbert}の構成を考えた。


    \section{小信号解析}
        前節では今回提案する構成によって乗算ができると考える理由を述べたが、本節では小信号解析により提案回路を用いたアナログ乗算が可能であることを示す。\par
        図\ref{fig:3_folded_mirror_gilbert}はギルバート乗算回路同様差動回路であるため差動半回路を考えることで回路全体の小信号解析を行うことができる。したがって半回路として考える部分を図\ref{fig:3_folded_mirror_gilbert_half}に示す。また、この時の小信号等価半回路を図\ref{fig:3_folded_mirror_half}に示す。
        \begin{figure}[!b]
            \centering
            \includegraphics[width=0.99\textwidth]{figures/chapter3/NtoNFolded_half.pdf}
            \caption{カレントミラーを組み合わせた折り返し型アナログ乗算回路の半回路として考える部分}
            \label{fig:3_folded_mirror_gilbert_half}
        \end{figure}
        \begin{figure}[!b]
            \centering
            \includegraphics[width=0.99\textwidth]{figures/chapter3/NtoNHalfDiffEqual.pdf}
            \caption{カレントミラーを組み合わせた折り返し型アナログ乗算回路の小信号半回路}
            \label{fig:3_folded_mirror_half}
        \end{figure}
        \clearpage
        ここで$g_{mB},g_{mMB},g_{mMA},g_{mA}$はそれぞれ$\mathrm{M_{B},M_{MB},M_{MA},M_{A}}$のトランスコンダクタンスであり、$g_{dB},g_{dMB},g_{dMA},g_{dA}$は$\mathrm{M_{B},M_{MB},M_{MA},M_{A}}$のドレインソース間抵抗である。また$R_{L}$は負荷抵抗であり、$v_{AS},v_{BD}$はそれぞれ接点A,Bの電位とする。この時、接点BにKCLを用いると$v_{BD}$は
        \begin{align}
            0&=g_{mB}(-v_{in})+(g_{dB}+g_{dMB})v_{BD}+g_{mMB}v_{BD}     \notag\\
            v_{BD}&=\frac{g_{mB}}{ g_{mMB}+g_{dMA}+g_{dMB} }v_{in}      \notag
        \end{align}
        と表せる。ここで、$g_{mB}>>g_{dMA},g_{dMB}$を仮定すると
        \begin{align}
            v_{BD}\approx \frac{g_{mB}}{g_{mMB}}v_{in}      \label{eq:3_vbd}
        \end{align}
        と近似することができる。次に$i_{A1},i_{A2}$はそれぞれ
        \begin{align}
            i_{A1}&=(g_{mA}-\Delta g_{m})(-v_{AS})+g_{dA}(v_{outp}-v_{AS})      \label{eq:3_ia1}\\
            i_{A2}&=(g_{mA}+\Delta g_{m})(-v_{AS})+g_{dA}(v_{outm}-v_{AS})      \label{eq:3_ia2}   
        \end{align}
        であることを用いると、接点AについてのKCLを考えることで$v_{AS}$は
        \begin{align*}
            g_{mMA}v_{BD}+g_{dMA}v_{AS}&=i_{A1}+i_{A2}   \\
            &=(g_{mA}-\Delta g_{m})(-v_{AS})+g_{dA}(v_{outp}-v_{AS})    \\
            &\quad\quad\quad\quad +(g_{mA}+\Delta g_{m})(-v_{AS})+g_{dA}(v_{outm}-v_{AS})       \\
            &=-2g_{mA}v_{AS}-2g_{dA}v_{AS}  \\
            v_{AS} &= -\frac{ g_{mMA} }{ 2g_{mA}+2g_{dA}+g_{dMA} }v_{BD}
        \end{align*}
        と計算できる。さらに$g_{mMA}>>2g_{dA},g_{dMA}$を仮定すると
        \begin{align}
            V_{AS} \approx -\frac{g_{mMA}}{2g_{mA}} v_{BD}      \label{eq:3_vas}
        \end{align}
        の近似ができる。ここで図\ref{fig:3_folded_mirror_half}が差動回路であることに注意すると、\mbox{$i_{A3}=-i_{A2}$} \mbox{$v_{outp}=-v_{outm}$}の関係が成り立つので
        \begin{align}
            i_{outp}&=i_{A1}+i_{A3}     \notag\\
            &=i_{A1}-i_{A2}     \notag\\
            &=2\Delta g_{m}v_{AS}+2g_{dA}v_{outp}     \label{eq:3_ioutp}
        \end{align}
        と$v_{outp}$を用いて$i_{outp}$を表すことができた。接点OについてのKVLを考えると
        \begin{align}
            v_{outp}&=0-R_{L}i_{outp}       \notag\\
            &=-R_{L}\left( 2\Delta g_{m}v_{AS}+2g_{dA}v_{outp}  \right)       \notag\\
            &=-\frac{ 2R_{L} }{ 1+2R_{L}g_{dA} } \cdot \Delta g_{m}v_{AS}       \label{eq:3_voutp}    
        \end{align}
        であるので、出力電圧$v_{out}$を$v_{out}:=v_{outp}-v_{outm}$とすれば$v_{outp}=-v_{outm}$であることに留意すると
        \begin{align*}
            v_{out}&=2v_{outp}  \\
            &=-\frac{ 4R_{L} }{ 1+2R_{L}g_{dA} } \cdot \Delta g_{m}v_{AS}
        \end{align*}
        と計算できる。ここで式(\ref{eq:3_vbd})、(\ref{eq:3_vas})を代入すると
        \begin{align*}
            v_{out}=\frac{ g_{mB} }{ g_{mA} }\cdot\frac{ g_{mMA} }{ g_{mMB} }\cdot\frac{ 2R_{L} }{ 1+2R_{L} g_{dA}}\cdot\Delta g_{m}v_{in}
        \end{align*}
        である。さらに\ref{ch:gilbert_valiable_gm}の結論を用いれば、$\mathrm{M_{A}}$のトランスコンダクタンス係数を$K$としたとき$\Delta g_{m}=2KV_{CTRL}$なので
        \begin{align}
            v_{out}=\frac{ g_{mB} }{ g_{mA} }\cdot\frac{ g_{mMA} }{ g_{mMB} }\cdot\frac{ 4KR_{L} }{ 1+2R_{L} g_{dA}}\cdot V_{CTRL}\cdot v_{in}      \label{eq:3_vout}
        \end{align}
        と計算することができた。したがってカレントミラーを組み合わせた折り返し型乗算回路は入力電圧$v_{in}$と制御電圧$V_{CTRL}$に比例した出力を得ることができる。特にカレントミラーの形状比が同一である場合、$g_{mMA}=g_{mMB}$となるため、ギルバート乗算回路の出力と一致することが分かる。


    \section{出力範囲}
        次に、\ref{ch:2_range}と同様の方法で出力範囲を求めギルバート乗算回路と比較を行う。まず、適切に動作する条件をすべてのMOSFETが飽和領域で動作することするための式(\ref{eq:2_binding_conditions})を再掲すると
        \begin{subequations}
            \begin{empheq}[left={\empheqlbrace}]{align*}
                &V_{GS}-V_{th}<V_{DS}          \\
                &V_{th}<V_{GS}              
            \end{empheq}
        \end{subequations}
        であった。これを図\ref{fig:3_folded_mirror_gilbert}の各MOSFETに適用すると
        \begin{subequations}
            \begin{empheq}[left={M_{A}:\empheqlbrace}]{align}
                &V_{rbias}+V_{CTRL}-V_{AS}-V_{th}<V_{out-\frac{1}{2}}-V_{AS}        \\
                &V_{th}<V_{rbias}-V_{CTRL}-V_{AS}
            \end{empheq}        \label{eq:3_ma_binding}
        \end{subequations}
        \begin{subequations}
            \begin{empheq}[left={M_{MA}:\empheqlbrace}]{align}
                &V_{M}-V_{th}<V_{AS}        \\
                &V_{th}<V_{dd}-V_{U}
            \end{empheq}        \label{eq:3_mma_binding}
        \end{subequations}
        \begin{subequations}
            \begin{empheq}[left={M_{U}:\empheqlbrace}]{align}
                &V_{dd}+v_{in}-V_{BS}-V_{th}<V_{dd}-V_{BS}      \\
                &V_{th}<V_{dd}-V_{U}
            \end{empheq}        \label{eq:3_mu_binding}
        \end{subequations}
        \begin{subequations}
            \begin{empheq}[left={M_{B}:\empheqlbrace}]{align}
                &V_{lbias}+v_{in}-V_{BS}-V_{th}<V_{M}-V_{BS}        \\
                &V_{th}<V_{lbias}-v_{in}-V_{BS}
            \end{empheq}        \label{eq:3_mb_binding}
        \end{subequations}
        \begin{subequations}
            \begin{empheq}[left={M_{L}:s\empheqlbrace}]{align}
                &V_{L}-V_{th}<V_{BS}        \\
                &V_{th}<V_{L}
            \end{empheq}        \label{eq:3_ml_binding}
        \end{subequations}
        という5つの不等式が得られる。ただし、$\mathrm{M_{MA}}$に関してはダイオード接続になっているので常に飽和領域で動作する。まず$\mathrm{M_{L}}$について、式(\ref{eq:3_ml_binding}b)の両辺に$-V_{th}$を加えると
        \begin{align*}
            0<V_{L}-V_{th}
        \end{align*}
        となる。これと式(\ref{eq:3_ml_binding}a)と合わせると
        \begin{align}
            0<V_{BS}    \label{eq:3_vbs_range}
        \end{align}
        を得る。同様に$\mathrm{M_{A},M_{B},M_{U}}$についてもa式とb式をまとめると
        \begin{align}
            \mathrm{M_{A}}\;:\;&V_{AS}<V_{out}-\frac{1}{2}v_{out}-2V_{CTRL}        \label{eq:3_vas_upper}\\
            \mathrm{M_{B}}\;:\;&V_{BS}+2v_{in}<V_{M}                               \label{eq:3_vm_lower}\\
            \mathrm{M_{U}}\;:\;&V_{M}<V_{dd}                                       \label{eq:3_vm_upper}
        \end{align}
        と表すことができる。式(\ref{eq:3_vm_upper})に関しては、$V_{m}$が電源電圧内であれば飽和領域で動作するということなので今回はほとんど考慮する必要がない。次に式(\ref{eq:3_vas_upper})の両辺に$-V_{th}$を加えることで式(\ref{eq:3_mma_binding}a)の左辺と等しくなることに注意すると、式(\ref{eq:3_vm_lower})より
        \begin{align}
            V_{BS}+2v_{in}&<V_{out}-\frac{1}{2}v_{out}-2V_{CTRL}-V_{th}         \notag\\
            V_{BS}+2v_{in}+2V_{CTRL}+V_{th}&<V_{out}-\frac{1}{2}v_{out}         \notag\\
            \frac{1}{2}v_{out}&<V_{out}-(V_{BS}+2v_{in}+2V_{CTRL}-V_{th})        \label{eq:3_vout_lower}
        \end{align}
        を得る。また、出力電圧は電源電圧にも制限を受けるため
        \begin{align}
            V_{out}+\frac{1}{2}&<V_{dd}                                          \notag\\
            \frac{1}{2}v_{out}&<V_{dd}-V_{out}                                  \label{eq:3_vout_upper}
        \end{align}
        が必要となる。式(\ref{eq:3_vout_lower})、(\ref{eq:3_vout_upper})の辺々を足し合わせると
        \begin{align}
            v_{out}&<V_{dd}-(V_{BS}+2v_{in}+2V_{CTRL}-V_{th})                  \label{eq:3_vout_range}
        \end{align}
        と求められた。これはギルバート乗算回路の出力範囲、式(\ref{ch:2_range})と比較すると$V_{th}$だけ大きくなっていることが分かる。この出力振幅の拡大はカレントミラーによって生じており、参照側はドレインとゲートを短絡しているがコピー側では飽和領域で動作する条件からドレイン電位を参照側よりもしきい電圧$V_{th}$分下げられることに起因する。つまりカレントミラーを多数接続することでさらに$\mathrm{M_{A}}$のソース電位を下げられる可能性があるが、カレントミラーのドレインソース間電圧が小さくなるとドレイン電流が小さくなることが考えられる。


    \section{シミュレーションによる確認}
        
        出力範囲が理論上はカレントミラーの閾電圧$V_{th}$分広げられることを前節で示すことができたが、本節では$\mathrm{ROHM\;0.18\;\mu m\;Process}$においてギルバート乗算回路と比較しカレントミラーを組み合わせた折り返し型乗算回路の出力範囲が広がっていることを確認する。

        \subsection{素子値の設計}
            今回は入力電圧と制御電圧を共に$0.2\;\mathrm{V}$として設計を行う。また、$\mathrm{ROHM\;0.18\;\mu m\;Process}$ではトリプルウェル構造を作ることができるので、各MOSFETのバルク端子は各々のソースと短絡させた。これはしきい電圧を小さくし、設計を容易にするためである。\par
            大まかな方針として高速化をするためにドレインソース間電圧に依存してほしくない部分を除きチャネル長を最小寸法である$0.18\;\mu m$を用いることとした。まずはしきい電圧を推定するためにドレイン電流($I_{D}$)-ゲートソース間電圧($V_{GS}$)特性を調べた。ただし、ここの時のチャネル長は$1\;\mu m$とした。また、トランスコンダクタンスは式(\ref{eq:2_gm})を再掲すると
            \begin{align*}
                g_{m}=\frac{\partial I_{D}}{\partial V_{GS}}=K(V_{GS}-V_{th})
            \end{align*}
            であった。これは$V_{GS}$に関する1次式であり、最小二乗法による近似が容易である。そのためこのトランスコンダクタンスに近似した1次間数からしきい電圧を推定した。この時の$I_{D}-V_{GS}$特性、トランスコンダクタンスとその近似直線を図\ref{fig:3_vth_est}に示す。
            


%    \begin{figure}[!b]
%        \centering
%        \includegraphics[width=0.99\textwidth]{figures/chapter}
%        \caption{}
%        \label{fig:3_}
%    \end{figure}

%    \begin{subequations}
%        \begin{empheq}[left={\empheqlbrace}]{align}
%        \end{empheq}        \label{eq:}
%    \end{subequations}

\chapter{FSK変調方式による電力・データ同時伝送システム}
\section{原理ならびに構成}
2.2節において述べたとおり,極大電力周波数の近傍にあるZRFは2つ存在し,それらの周波数における出力電力$P_{ZRF}$は等しい.したがって,2値のデータを各ZRFに割り当てることにより,データによる出力電力の変動を避けながら,電力とデータの同時伝送(Simultaneous Wireless Information and Power Transfer : SWIPT)が実現できると考えられる.これは,2値データに対応させて電力伝送に用いる交番磁界の周波数を切り替える,すなわち周波数偏移(Frequency Shift Keying : FSK)変調を行うことと同義である.実際にこの動作を行うためには,図\ref{concept}のように,$f_{ZRF1}$あるいは$f_{ZRF2}$のいずれかのみにロックするPLLを用意し,データに対応させてそれらの出力を選択すればよい.

\begin{figure}[b]
\begin{center}

\includegraphics[width=80mm]{figures/concept.pdf}
\caption{FSK変調方式による電力・データ同時伝送システムの概念図}
\label{concept}
\end{center}

\end{figure}

図\ref{concept}の構成でFSK変調を実現するためには,各PLLがノイズや回路の初期状態によらず常に同じZRFにロックしている必要がある.図\ref{entireblock}の回路においては,PLLが複数あるうちのいずれのZRFにロックするかは考慮されておらず,上述した要因によりロックする周波数が変動してしまう問題があった.この問題を避けるためには,何らかの方法によりVCOの出力周波数範囲を制限すればよい.例えば,図\ref{entireblock}において用いているPLL-IC CD74HC4046A(Texas Instruments)では,2つの外付け抵抗によりVCOの出力周波数範囲を設定することができる\cite{4046}.極大電力周波数の近傍にあるZRF$f_{ZRF1}, f_{ZRF2}$の範囲は,MR-WPT回路をあらかじめ数値計算することにより求められるから,$f_{ZRF1}$のみを追従する回路,あるいは$f_{ZRF2}$のみを追従する回路をそれぞれ構成することができる.\par
図\ref{fskentirecircuit}に,提案するFSK変調方式による電力・データ同時伝送(FSK-SWIPT)システムの全体構成を示す.送電側は,図\ref{entireblock}におけるPLLをFSK変調器に置き換えた構成である.また受電側は,負荷抵抗$R_L$に直流出力電圧$V_{OUT}$を供給するためのブリッジダイオードによる全波整流回路と,リミタ回路,コンパレータを用いた波形整形回路ならびに復調用のFPGAから成る.変復調回路の詳細については後述する.

\begin{figure}[h]
\begin{center}

\includegraphics[width=160mm]{figures/fskentirecircuit.pdf}
\caption{FSK-SWIPTシステムの全体構成図}
\label{fskentirecircuit}
\end{center}

\end{figure}

\section{FSK変調回路}
\subsection{全体構成}
図\ref{fskmodulator}に,設計したFSK変調回路の全体構成を示す.同回路は,PFD1,LPF1,ならびにVCO1から成り$f_{ZRF1}$にロックするPLL1と,同様の構成で$f_{ZRF2}$にロックするPLL2を主体として構成されている.これらの回路の出力を,マルチプレクサによりデータ信号に対応させて切り替えることによりFSK変調を行う.また,補助的な回路として,タイミング制御回路(Timing Controller),ロック検出回路(Lock Detector)ならびに電圧制限回路(Voltage Restrictor)を付加している.以下,順に述べる.

\begin{figure}[h]
\begin{center}

\includegraphics[width=160mm]{figures/fskmodulator.pdf}
\caption{FSK変調回路}
\label{fskmodulator}
\end{center}

\end{figure}

\subsection{内部回路}
\subsubsection{タイミング制御回路}
MR-WPT回路がZRFで駆動されているとき,LC共振回路のスイッチング特性から,回路を流れる電流$i_1$は正弦波状になる.3.2節において述べたとおり,$i_1$はコンパレータにより矩形波電圧信号に変換され,$i_1$が正弦波状であればそのDuty比はほぼ50\%になる.一方,MR-WPT回路がZRFでない周波数で駆動されているとき,$i_1$は正弦波状ではなくなるため,矩形波電圧信号のDuty比も不規則に変化する.このような信号をPFDに入力した場合,適切な位相比較が行えず,その出力が不安定になる.\par
提案手法によりFSK変調を行う場合,データの遷移に対応して回路を駆動する周波数を切り替える.周波数を切り替えるとき,電流$i_{1}$の周波数$f$は,過渡的に2つのZRFの間の周波数($f_{ZRF1}<f<f_{ZRF2}$)となる.このとき,先述した理由により,PFDの出力が不安定になり,ひいてはVCOの出力も不安定な状態となる.これを避けるためには,データの遷移時にPFDの出力を遮断すれば良い.このような動作を実現するために,タイミング制御回路を挿入する.PFDの出力が遮断されているとき,各VCOの制御電圧はキャパシタ$C_1,C_2$により一定に保持される.\par 
また,タイミング制御回路は,上記の動作に加え,MR-WPT回路がいずれかのPLLの出力で駆動されているとき他方のPLL中のPFD出力を遮断するよう動作する.例えば,MR-WPT回路がPLL1で駆動されているとき,PFD2の出力は遮断される.これは,各PLLを独立して動作させることで,ロック時間の短縮やループの安定化を図るためである.\par 
すなわち,タイミング制御回路は,データに対応して図\ref{fskmodulator}における電圧制御スイッチ$S_1, S_3$を開閉するとともに,データの遷移時には両方のスイッチを開放するような動作をする.これは,3.2.2節で述べた,スイッチング信号の遷移時にハーフブリッジにおける2つのFETをいずれもターンオフさせる,デッドタイム生成回路と全く同じ動作である.よって,回路構成は図\ref{deadtimegenerator}と同一のものであり,同図における「VCO output sig.」を「DATA signal」に,「High/Low-side drive sig.」をスイッチ$S_1, S_3$の開閉信号にそれぞれ読み替えれば良い.

\subsubsection{ロック検出回路}
PLLのロック時間とVCOの位相雑音特性は,LPFの時定数$\tau$に大きく依存する.ロック時間を短くしたい場合は$\tau$を小さくすればよいが,その場合LPFの遮断周波数が高くなってしまうため,PFDの出力パルスの高調波成分を十分に除去することができず,結果としてVCOの出力位相雑音が増大する.したがって,一般にロック時間と位相雑音とはトレードオフの関係にある.これを改善するためには,入力信号の位相差$|\theta_1-\theta_2|$が大きい,すなわちPLLがロック状態から遠いときは$\tau$を小さく,$|\theta_1-\theta_2|$が小さいときは$\tau$を大きくすればよい.\par 
Lock Detector回路は,入力信号の位相差$|\theta_1-\theta_2|$があるしきい値$\theta_t$より大きいか否かを判定し,図\ref{fskmodulator}における各LPF中にあるスイッチ$S_2,S_4$を制御する回路である.図\ref{lockdetector}のように,PFDで2信号の位相差を検出し,その平均電圧と基準電圧$V_{REF}$とを比較して$S_2,S_4$を開閉する.$\theta_t$は$V_{REF}$ならびに$R$を可変することで実験的に調整する.各スイッチが閉じているときは,開いているときよりも$\tau$が小さくなる.この回路を用いることで,ロック時間を短縮し,また外来ノイズ等による突発的なロック外れを抑制する.

\begin{figure}[h]
\begin{center}

\includegraphics[width=120mm]{figures/lockdetector.pdf}
\caption{ロック検出回路}
\label{lockdetector}
\end{center}

\end{figure}

\subsubsection{電圧制限回路}
図\ref{fskentirecircuit}の回路で用いているPLL-IC  CD74HC4046(Texas Instruments)のVCOは,制御電圧-発振周波数特性がほぼ線形である領域と,指数関数的に非線形に変化する領域とが存在する.具体的には,ICを電源電圧$5 \, \mathrm{V}$で使用した場合,制御電圧がおよそ$1-4 \, \mathrm{V}$程度の範囲で制御電圧-発振周波数特性はほぼ線形となる\cite{Enzaka2014}.図\ref{fskentirecircuit}の回路では,FSK変調のためにVCOの発振周波数を適切に制限しなければならず,したがってVCOの制御電圧-発振周波数特性がほぼ線形である領域でVCOを駆動する必要がある.\par 
図\ref{voltagerestrictor}は,$0-V_{DD} \, \mathrm{[V]}$の値をとるLPFの出力電圧$v_{LPFout}$に対して,
\begin{eqnarray}
v_{vcoin}=\left\{ \begin{array}{ll}
v_y & (v_{LPFout}>v_y) \\
v_{LPFout} & (v_x<v_{LPFout}<v_y) \\
v_x & v_{LPFout}<v_x \\
\end{array} \right.
\end{eqnarray}
のように$v_x<v_{vcoin}<v_y$の範囲に制限されたVCOの制御電圧$v_{vcoin}$を出力する回路である.ここで,ダイオードは理想素子として扱っている.例えば,$v_{LPFout}>v_y$のとき,ダイオード$D_2$が導通するから,$v_{vcoin}$の上限は$v_y$に制限される.同様にして,$v_{LPFout}$の下限も$v_x$の制限される.実際は,$v_{vcoin}$の上限値と下限値はダイオード$D_1, \, D_2$の順方向電圧降下等に影響されるため,それらの値は$R_2$ならびに$R_3$を可変することにより実験的に調整する.


\begin{figure}[b]
\begin{center}

\includegraphics[width=110mm]{figures/voltagerestrictor.pdf}
\caption{電圧制限回路}
\label{voltagerestrictor}
\end{center}

\end{figure}


\section{シミュレーション}

図\ref{fskentirecircuit}ならびに図\ref{fskmodulator}の回路について,LTspiceでシミュレーションを行った.シミュレーション回路を図\ref{fsksimulationcircuit}に示す.ここで,伝送部のパラメータは表1のとおりとし,データ信号は周波数$5 \, \mathrm{kHz}$,Duty比0.5の矩形波(0101…の連続信号)とした.また,VCO,電圧制御スイッチとバッファには理想素子を用い,受信側の整流回路は等価的な負荷抵抗$R_{L}=5 \, \mathrm{\Omega}$に置換し,FPGAによる復調回路は省略した.さらに,PFDならびにハーフブリッジ回路は図\ref{pllsimulationcircuit}と同様の構成のものをサブサーキット化している.\par 

図\ref{fsksimulationcircuit}におけるVCO1, VCO2は,$0-1 \, \mathrm{V}$の制御電圧に対して1次関数で表される周波数を出力する理想VCOであり,出力周波数範囲はそれぞれ$500-700 \, \mathrm{kHz}$, $730-1100 \, \mathrm{kHz}$に設定してある.したがって,各VCOの制御電圧$v_{vcoin1}$,$v_{vcoin2} \, \mathrm{[V]}$と出力周波数$f_{vcoin1}$,$f_{vcoin2} \, \mathrm{[kHz]}$の関係は,
\begin{align}
f_{vcoin1} &=500+200\cdot v_{vcoin1} \\
f_{vcoin2} &=730+370\cdot v_{vcoin2}
\end{align}
\begin{figure}[H]
\begin{center}

\includegraphics[width=155mm]{figures/fsksimulationcircuit.pdf}
\caption{FSK-SWIPTシステムのシミュレーション回路}
\label{fsksimulationcircuit}
\end{center}

\end{figure}
で表される.2つのZRFの周波数の理論値は,式(2.17),(2.18)から,$f_{ZRF1} \simeq 597\mathrm{kHz}, \, f_{ZRF2} \simeq 979 \mathrm{kHz}$であり,それらに対応するVCO1, VCO2の制御電圧$v_{vcoin1}, \, v_{vcoin2}$は,式(4.2),(4.3)より$v_{vcoin1}\simeq 0.49, \, v_{vcoin2} \simeq 0.67 \, \mathrm{V}$である.図\ref{fsksimulationgraph1}に,$v_{vcoin1}, \, v_{vcoin2}$の過渡解析結果を示す.同図から明らかなように,VCOの制御電圧はほぼ理論値付近にロックしている.データ信号の周期は$0.2 \, \mathrm{ms}$,シミュレーション時間は$1 \, \mathrm{ms}$であるから,データの値によらず安定した制御電圧が得られている.\par
図\ref{fsksimulationgraph2}は,データ信号ならび負荷抵抗$R_L$の電流の過渡解析結果である.定常状態においては,データによらず負荷電流の振幅はほぼ等しくなっている.また,データのHIGHとLOWに対応して周波数が変化している.図\ref{fsksimulationgraph3}は,電流$i_1$の過渡解析結果をFFT解析することにより得られたスペクトルである.2つのZRFの理論値$f_{ZRF1} \simeq 597\mathrm{kHz}, \, f_{ZRF2} \simeq 979 \mathrm{kHz}$付近にスペクトルのピークがあり,シミュレーション回路が所望の動作を実現できているといえる.
\begin{figure}[H]
\begin{center}

\includegraphics[width=130mm]{figures/fsksimulationgraph1.pdf}
\caption{VCO制御電圧の過渡解析結果}
\label{fsksimulationgraph1}

\vspace{3mm}

\includegraphics[width=130mm]{figures/fsksimulationgraph2.pdf}
\caption{データ信号(上段)ならびに負荷電流(下段)の過渡解析結果}
\label{fsksimulationgraph2}

\end{center}
\end{figure}

\begin{figure}[!h]
\begin{center}

\includegraphics[width=140mm]{figures/fsksimulationgraph3.pdf}
\caption{電流$i_1$のスペクトル}
\label{fsksimulationgraph3}

\end{center}
\end{figure}
\section{FPGAを用いたFSK復調器}
\subsection{FPGAボードの概要}

\begin{figure}[t]
\begin{center}

\includegraphics[width=140mm]{figures/fpgaphoto.pdf}
\caption{FPGAボード Zybo Z7-20}
\label{fpgaphoto}

\end{center}
\end{figure}

本研究では,FSK信号の復調用として,図\ref{fpgaphoto}に示すXilinx社のZybo Z7-20(以下Zybo)を使用した.以下,その概要について述べる.\par 
Zyboは,メインFPGAチップであるXilin社のZynq-7020のほか,入出力スイッチ,microSDカードスロット,HDMI端子,オーディオインターフェースなどを備えている.また,ユーザーが任意に書き換えられるロジック(Programmable Logic : PL)のほか,CPU(Processing System : PS)が搭載されている.リアルタイム性が求められる処理はPL部で,他の処理はPS部で行うことにより,高度な処理が実現できる.また,PS部にはLinuxなどのOSを搭載することもでき,Raspberry Piのような使い方をすることも可能である.PL部の記述には,ハードウェア記述言語であるVerilog-HDL\cite{Kobayashi2018,Kimura2009,Kimura2001}またはVHDLを用いる.本研究においてはVerilog-HDLを採用した.なお,PS部の記述は,通常のマイコンと同じようにCやC++などの言語を用いる.また,開発環境はPL部とPS部でそれぞれ別のものが用意されており,PL部は「Vivado」,PS部は「Vitus」というソフトを使う必要がある.\par
ZyboのPS部には$33.3333 \, \mathrm{MHz}$のクロックが供給されている\cite{zybo}.PSの内部にはPLLが搭載されており,この$33.3333 \, \mathrm{MHz}$のクロックを元として,最大4系統,$667\, \mathrm{MHz}$の動作クロックを得ることができる.動作クロックの周波数は,ユーザが任意に指定することができるが,内部PLLの分周器の構成による制限から,指定した周波数と差異が生じることもあり注意が必要である.また,PL部の動作クロックは,PS部のPLLが生成したクロックを用いることができるほか,PL部専用に独立して搭載された$125 \, \mathrm{MHz}$の信号源を利用することができる.本研究では,PS部を使用せずPL部のみで復調器を構成したため,この$125 \, \mathrm{MHz}$のクロックを用いた.\par 

\subsection{FSK復調器の動作原理}
FSK信号の復調方法は,フィルタ法\cite{Araki1985},零交差検波法\cite{Araki1985},PLL検波法\cite{Yanagisawa1998},2次相関器による方法\cite{Gardner1996},2カウンタ法\cite{Sankar2017}など様々なものが提案されている.本研究では,実装ならびに設計変更の容易性,ビットレート,外部インターフェースとの接続を見据えた拡張性などを勘案して,FPGAを用いてカウンタを構成して復調する手法を採用した.以下,その原理について述べる.\par 
FSK信号を復調する最も簡明な方法は,その周波数あるいは周期を測定することであるが,これらを正確かつ高速に行うことは困難である.そこで,本研究においては,FSK信号の半周期におけるFPGAの動作クロックの立ち上がりエッジを計数することにより,間接的に周期を測定する手法を採用した.例として,データが1のときのFSK信号の周波数$f_{mark}=979 \, \mathrm{kHz}$, 0のときのFSK信号の周波数$f_{space}=598 \, \mathrm{kHz}$,FPGAのクロック周波数$f_{clk}=125 \, \mathrm{MHz}$の場合を考える.FSK信号の一周期における$f_{clk}$の立ち上がりエッジを計数するカウンタを構成した場合,FSK信号の周波数が$f_{mark}$のときの計数値$N_{mark}$,$f_{space}$の時の計数値$N_{space}$は,それぞれ
\begin{align}
N_{mark} &=\frac{125 \times 10^6}{979 \times 10^3}\simeq 128 \\
N_{space} &=\frac{125 \times 10^6}{598 \times 10^3} \simeq 209
\end{align}
となる.したがって,FPGA内部にカウンタを構成し,その値が適当なしきい値$N_{th}$より大きいか否かを判定することで,データを復調することができる.\par 
実際のFSK-SWIPTシステムにおいては,$f_{space}, \, f_{mark}$はそれぞれ2つのZRF$f_{ZRF1}, \, f_{ZRF2}$に対応し,2.2節における解析で述べたように$f_{ZRF1}<f_{ZRF0}<f_{ZRF2}$である.したがって,データ判定のしきい値$N_{th}$は$f_{ZRF0}$を基準にして決定すればよい.表1の数値例においては,式(2.12)より$f_{ZRF0} \simeq712 \, \mathrm{kHz}$であるから,
\begin{equation}
N_{th} =\frac{125 \times 10^6}{712 \times 10^3} \times \simeq 176
\end{equation}
となる.

\subsection{FSK単体でのFSK変復調器の動作実験}
前項で述べた手法による復調器が正しく動作するか,FPGA内部にFSK変調器と復調器を併せて実装して確認した.動作確認は,論理シミュレーションならびにFPGA内部にロジックアナライザを構成して波形を測定することにより行った.ロジックアナライザの構成は,Xilinx社の開発ソフトVivadoに標準搭載されている機能であり,外部の測定器を使わずに動作波形を測定することができデバッグに非常に有用である\cite{Kobayashi2018}.また,FSK変復調器ならびにそのテストベンチのVerilog-HDLソースコードを付録Bに添付する.\par
FPGA内部のFSK変調器は,$125 \, \mathrm{MHz}$のクロックを適当に分周して$f_{mark} \simeq 979 \, \mathrm{kHz}$, $f_{space} \simeq 598 \, \mathrm{kHz}$の2つの信号を生成し,それらの信号をデータに対応させて切り替えることによりFSK信号を得る構成とした.データ信号は,クロックを分周して得た,約$57.6 \, \mathrm{kHz}$(115.2kbps相当)の矩形波(101010…の連続信号)とした.\par 
図\ref{logicanalyzer}に,ロジックアナライザにより測定したFSK変復調器の動作波形を示す.同図上段がデータ信号,中段がFSK信号,下段が復調されたデータ信号である.同図より,復調データは元データ信号に対して遅延が生じているものの,正しく復調出来ていることが分かる.図\ref{counter}は,FSK信号の一周期におけるクロックの立ち上がりエッジを計数するカウンタの値の遷移を示したものである.カウンタの値は,データが遷移するときを除き約210回と130回の2値となっており,これは式(4.4),(4.5)の計算結果と一致していることから,カウンタが所望の動作を実現できているといえる.

\begin{figure}[b]
\begin{center}

\includegraphics[width=160mm]{figures/logicanalyzer.png}
\caption{FSK変復調器の動作波形}
\label{logicanalyzer}

\end{center}
\end{figure}

\begin{figure}[h]
\begin{center}

\includegraphics[width=100mm]{figures/counter.pdf}
\caption{カウンタの値の時間遷移}
\label{counter}

\end{center}
\end{figure}

\section{実装ならびに実験}
\subsection{実装回路}
本研究では,提案回路をプリント基板上(Printed Circuit Board : PCB)に実装し,各種特性の測定実験を行った.図\ref{circuitphoto}に,実装した回路の写真を示す.また,設計した回路の全体回路図ならびにPCBのレイアウト図を付録Aに添付する.実装回路の主要なICの型番ならびに回路定数を表4.1に示す.

\begin{table}[h]
\centering
\caption{主要なICならびに回路定数 (定数は100 kHzにて測定)}
\begin{tabular}{c|c|c}
\hline
\multicolumn{1}{c|}{素子} & \multicolumn{1}{c|}{型番/定数} & \multicolumn{1}{c}{メーカー} \\ \hline
PLL & CD74HC4046 & Texas Instruments \\ \hline
ハーフブリッジドライバ & ADuM3223 & Analog Devices\\ \hline
ハーフブリッジ用NchMOSFET & RD3L220SN & Rohm \\ \hline
送電コイル$L_1$ & $3.86\, \mathrm{\mu H}$, \, ESR : $0.053\, \mathrm{\Omega}$ & - \\ \hline
受電コイル$L_2$ & $3.91\, \mathrm{\mu H}$, \, ESR : $0.034\, \mathrm{\Omega}$ & - \\ \hline
送電側共振キャパシタ$C_1$ & $10.05\, \mathrm{n F}$, \, ESR : $0.043\, \mathrm{\Omega}$ & - \\ \hline
受電側共振キャパシタ$C_2$ & $10.07\, \mathrm{n F}$, \, ESR : $0.052\, \mathrm{\Omega}$ & - \\ \hline
\end{tabular}
\end{table}

\begin{figure}[p]

	\begin{center}
    \subfloat[送電側]{
    \includegraphics[width=150mm]{figures/transmitterphoto.png}
    }
    \\ \vspace{5mm}
    \subfloat[受電側]{
    \includegraphics[width=130mm]{figures/receiverphoto.png}
    }
	\end{center}
    
  \caption{実装回路}\label{circuitphoto}
\end{figure}

\subsection{実験}
実装回路の特性を調査するため,以下の7種の実験を行った.本項では,それらの実験の方法ならびに結果について順に述べる.実験に使用した測定器を表4.2に示す.

\begin{itemize} \setlength{\itemsep}{-0.2cm}
\item コイル間距離-結合係数特性の測定
\item ZRFの手動追従実験
\item ZRFの自動追従実験
\item データ伝送時の出力スペクトルの測定
\item データ伝送速度-電力効率特性の測定
\item 文字列データ伝送実験
\item 符号誤り率(Bit Error Rate : BER)の測定
\end{itemize}

\begin{table}[h]
\centering
\caption{測定機器}
\begin{tabular}{c|c|c}
\hline
\multicolumn{1}{c|}{名称} & \multicolumn{1}{c|}{型番} & \multicolumn{1}{c}{メーカー} \\ \hline
高周波電圧計(True RMS) & 3400A & Hewlett-Packard \\ \hline
周波数カウンタ & 5535A & Hewlett-Packard \\ \hline
スペクトラムアナライザ & 8561A & Hewlett-Packard \\ \hline
直流電源装置 & KPS3010D & Wanptek\\ \hline
信号発生器 & MHS-5200A & KKmoon \\ \hline
BER測定器 & ME448B & Anritsu \\ \hline
\end{tabular}
\end{table}

\subsubsection{コイル間距離-結合係数特性の測定}
はじめに予備実験として,送受電コイル間の距離$d$と結合係数$k$の測定を行った.このときの測定系を図\ref{dvsk}に示す.同図において,$L_1, L_2$はそれぞれ送受電コイル,$r_{L1},r_{L2}$はそれらのESRである.また,本実験を含め,本稿で報告する実験で用いた高周波電圧計はすべて真の実効値を指示するものである.本実験の測定手順を以下に示す.

\begin{enumerate}
  \item 信号発生器から周波数$1 \, \mathrm{MHz}$の正弦波を出力し,高周波電圧計の指示値$v_1$が$1 \, \mathrm{V}$となるよう信号発生器の出力振幅を調整した.
  \item 信号発生器の出力を保ちながら,高周波電圧計を受電側に付け替え,コイル間の距離$d$を$5 \, \mathrm{mm}$から$13 \, \mathrm{mm}$まで$1 \, \mathrm{mm}$ずつ変化させたときの受電コイルの電圧$v_{out}$を測定した.送受電コイルは,距離の測定を容易にするため,図\ref{coilphoto}のように方眼紙上で相対させた.
  \item 同様の測定を3回行い,各距離における受電コイルの電圧$v_{out}$の平均$v_{outave}$を求めた.
  \item 次の近似式
  \begin{align}
	k \simeq \sqrt{\frac{L_1}{L_2}} \cdot \frac{v_{outave}}{v_1} =\sqrt{\frac{L_1}{L_2}} \cdot v_{outave}
  \end{align}
を用いて,各距離における結合係数$k$を求めた.
\end{enumerate}
距離$d$と結合係数$k$の測定結果を図\ref{dvskgraph}に示す.なお,今回は$k$がある程度大きい領域($k>0.3$程度)での実験を企図していたため,測定は$d=13 \, \mathrm{mm}$で打ち切った.

\begin{figure}[h]
\begin{center}
\includegraphics[width=160mm]{figures/dvsk.pdf}
\caption{コイル間距離-結合係数特性の測定系}
\label{dvsk}
\end{center}
\end{figure}

\begin{figure}[h]
  \centering
    \begin{tabular}{c}
       \begin{minipage}{0.50\hsize}
        \centering
          \includegraphics[width=75mm]{figures/coilphoto.jpg}
                          \caption{結合係数特性の測定風景}
                          \label{coilphoto}
      \end{minipage}
  \hspace{5mm}
      \begin{minipage}{0.50\hsize}
        \centering
          \includegraphics[width=75mm]{figures/dvskgraph.pdf}
                          \caption{コイル間距離-結合係数特性の測定結果}
						  \label{dvskgraph}
      \end{minipage} \\
     \end{tabular}
\end{figure}  


\subsubsection{ZRFの手動追従}
PLLによる自動追従動作を確認する前に,先ずはZRFを手動で追従することにより,各コイル間距離$d$における2つのZRF$f_{ZRF1},f_{ZRF2} \, (f_{ZRF1}<f_{ZRF2})$と,各ZRFにおける電力効率を測定した.このときの測定系を図\ref{manual}に示す.同図において,$C_1, C_2$はそれぞれ送受電側の共振キャパシタ,$r_{1},r_{2}$はそれぞれ$L_1$と$C_1$,$L_2$と$C_2$の合成ESRであり,値はLCRメータを用いて周波数$100 \, \mathrm{kHz}$で測定したものである.また,$R_L$は負荷抵抗である.本実験の測定手順を以下に示す.

\begin{enumerate}
  \item コイル間距離$d =5 \, \mathrm{mm}$, ハーフブリッジの入力電圧$V_{in}=5 \, \mathrm{V}$とした.
  \item 位相比較器の出力$V_{PD}$が極小となる(i.e. スイッチング電圧とスイッチング電流の位相差が最小になる)ように信号発生器の出力周波数$f$を調整した.$V_{PD}$が極小となる周波数のうち,最低のものを$f_{ZRF1}$,最高のものを$f_{ZRF2}$として記録した.
  \item 各ZRFにおける直流電源装置の出力電流$I_{in}$ならびに受電側の負荷電圧$v_{out}$を記録し,電力効率$\eta$を
  \begin{align}
  \eta=\frac{v_{out}^2/R_L}{V_{in}I_{in}}
  \end{align}
  として求めた.
  \item コイル間距離を$1 \, \mathrm{mm}$ずつ$13 \, \mathrm {mm}$まで変化させ,同様の測定を行った.
\end{enumerate}
なお,本実験の結果については次項でまとめて述べる.

\subsubsection{PLLによるZRFの自動追従}
次に,マルチプレクサの出力をドライバ回路に接続し,ZRFをPLLにより自動追従させる実験を行った.このときの測定系を図\ref{auto}に示す.実験手順は次のとおりである.

\begin{enumerate}
  \item コイル間距離$d =5 \, \mathrm{mm}$, ハーフブリッジの入力電圧$V_{in}=5 \, \mathrm{V}$とした.
  \item データ入力をHIGH(電源電圧レベル)とした.このとき,マルチプレクサによりPLL1の出力が選択され,回路が正しく動作すればドライバ回路の出力周波数$f_{sw}$は$f_{ZRF2}$でロックされる.
  \item ドライバ回路の出力周波数$f_{sw}$を周波数カウンタで測定し,$f_{ZRF2}$として記録した.また,直流電源装置の出力電流$I_{in}$ならびに受電側の負荷電圧$v_{out}$を記録し,電力効率$\eta$を求めた.
  \item コイル間距離を$1 \, \mathrm{mm}$ずつ$13 \, \mathrm {mm}$まで変化させ,同様の測定を行った.
  \item データ入力をLOW(GNDレベル)として,$f_{sw}$が$f_{ZRF1}$でロックされるようにした上で,以上の測定を繰り返した.
  \item ここまでの測定を計3回行い,その平均値を結果とした.
\end{enumerate}
図\ref{tracking1}に,手動/自動追従によるZRFの測定結果ならびに理論曲線を示す.同図中において,実線が理論曲線,▲印が手動追従,●印が自動追従の測定結果である.なお,測定はコイル間距離$d$をパラメータとして行ったが,図では前々項における実験の結果を用いて$k$をパラメータとしている.同図より,手動/自動追従におけるスイッチング周波数$f$はほぼ等しく,自動追従回路が正しく動作しているものと考えられる.\par
図\ref{tracking2}は,手動/自動追従させた場合の各ZRFにおける電力効率の測定結果である.各ZRFにおける電力効率は,手動の場合と自動の場合とで概ね等しいことがわかる.また,結合係数$k$に対してほぼフラットな特性が得られている.2つのZRFにおける電力効率の差(i.e. 出力電力の差)はおおむね3\%程度であり,これは一次側共振回路と二次側共振回路の共振周波数が等しい($L_1C_1=L_2C_2$)のときに,2つのZRFにおける出力電力が等しくなるという解析結果と対応するものと考えられる(現実的には$L_1C_1=L_2C_2$とならないため,3\%程度の差が生じたと推察される).

\vspace{1cm}

\begin{figure}[h]
\begin{center}
\includegraphics[width=160mm]{figures/manual.pdf}
\caption{ZRFの手動追従実験の測定系}
\label{manual}
\vspace{1cm}

\includegraphics[width=160mm]{figures/auto.pdf}
\caption{ZRFの自動追従実験の測定系}
\label{auto}

\end{center}
\end{figure}

\begin{figure}[p]
\begin{center}
\includegraphics[width=140mm]{figures/tracking1.pdf}
\caption{ZRFの測定結果}
\label{tracking1}
\vspace{1cm}

\includegraphics[width=140mm]{figures/tracking2.pdf}
\caption{電力効率の測定結果}
\label{tracking2}

\end{center}
\end{figure}

\subsubsection{データ伝送時の出力スペクトルの測定}
前項の実験により,データ入力を固定したときはZRFの追従動作が行えることが確認できた.次に,データとして矩形波信号を入力したときの出力スペクトルを測定することにより,どの程度のビットレートまで追従動作が行えるかを調査した.このときの測定系を図\ref{spectrum}に示す.また,測定の様子の写真を図\ref{spectrumphoto}に示す.実験手順は次のとおりである.

\begin{enumerate}
  \item コイル間距離$d =5 \, \mathrm{mm}$, ハーフブリッジの入力電圧$V_{in}=5 \, \mathrm{V}$とした.
  \item 送電基板のデータ入力端子に,データ信号として周波数$f_{DATA}=0.1 \, \mathrm{kHz}$の矩形波信号を入力し,そのときの出力電圧$v_{out}$のスペクトルをスペクトラムアナライザで観測した.
  \item スペクトラムアナライザのカーソル機能を用い,スペクトルに現れた2つのピーク周波数を$f_{ZRF1},f_{ZRF2}$として記録した.
  \item 周波数$f_{DATA}$を徐々に上げながら,同様の測定を行った.
  \item コイル間距離$d =10 \, \mathrm{mm}$として,以上の測定を繰り返した.
\end{enumerate}
図\ref{spectrumgraph} (a),(b)に,各コイル間距離における$f_{DATA}$と$f_{ZRF1},f_{ZRF2}$の測定結果を示す.同図において,プロットが無い箇所はピークが判別できず欠測したものである(欠測があるのは$20 \mathrm{kHz}$以上の測定点であるが,このときは$10 \mathrm{kHz}$ごとに測定している).なお参考として,$f_{DATA}=0.1 \, \mathrm{kHz}$,$d=10 \, \mathrm{mm}$のときのスペクトルを図\ref{spectrumimage}に示す.\par
図\ref{spectrumgraph}より,$d=10 \, \mathrm{mm}$のときは$80 \, \mathrm{kHz}$程度,$d=5 \, \mathrm{mm}$のときは$10 \, \mathrm{kHz}$程度までZRFの追従動作が行えていると考えられる.また,$d=5 \, \mathrm{mm}$のとき,$30-60 \, \mathrm{kHz}$で欠測したのち,より高い周波数では再びピークが観測される現象が確認された.

\begin{figure}[p]
\begin{center}

\includegraphics[width=160mm]{figures/spectrum.pdf}
  \caption{出力スペクトルの測定系}
  \label{spectrum}
\vspace{2cm}

\includegraphics[width=160mm]{figures/spectrumphoto.jpg}
  \caption{出力スペクトルの測定風景}
  \label{spectrumphoto}
  \end{center}
\end{figure}

\begin{figure}[p]
\begin{center}

    \subfloat[$d= 5\, \mathrm{mm}$]{
    \includegraphics[width=75mm]{figures/spectrumgraph5.pdf}
    }    
    \subfloat[$d= 10\, \mathrm{mm}$]{
    \includegraphics[width=75mm]{figures/spectrumgraph10.pdf}
    }
    
  \caption{出力スペクトルのピーク周波数の測定結果}\label{spectrumgraph}
  
  \vspace{2cm}
  
  \includegraphics[width=160mm]{figures/spectrumimage.jpg}
  \caption{ $f_{DATA}=0.1 \, \mathrm{kHz}$,$d=10 \, \mathrm{mm}$のときの出力スペクトル}
  \label{spectrumimage}
  
  \end{center}
\end{figure}


\subsubsection{データ伝送速度-電力効率特性の測定}
データ伝送速度が電力効率に与える影響を調べる実験である.このときの測定系は,図\ref{spectrum}におけるスペクトラムアナライザを高周波電圧計に置き換えたものである.実験手順を以下に示す.

\begin{enumerate} \setlength{\itemsep}{-0.2cm}
  \item コイル間距離$d =5 \, \mathrm{mm}$, ハーフブリッジの入力電圧$V_{in}=5 \, \mathrm{V}$とした.
  \item 送電基板のデータ入力端子に,データ信号として周波数$f_{DATA}=0.1 \, \mathrm{kHz}$の矩形波信号を入力し,そのときの入力電流$I_{in}$ならびに出力電圧$v_{out}$を測定した.
  \item 式(4.8)により,電力効率$\eta$を求めた.
  \item 周波数$f_{DATA}$を徐々に上げながら,同様の測定を行った.
  \item コイル間距離$d =10 \, \mathrm{mm}$として,以上の測定を繰り返した.
\end{enumerate}
図\ref{datavseta} (a),(b)に,各コイル間距離における伝送速度-電力効率特性の測定結果を示す.なお,$f_{DATA}$を高くすると出力電圧が大きく振れ不安定になる現象が現れたため,$d =10 \, \mathrm{mm}$のときは$f_{DATA}=80 \, \mathrm{kHz}$,$d =5 \, \mathrm{mm}$のときは$f_{DATA}=10 \, \mathrm{kHz}$で,それぞれ測定を打ち切った.電力効率は,$f_{DATA}$に対してほぼフラットな特性,あるいは$f_{DATA}$が高くなるとともに上昇する現象が認められる.

\begin{figure}[h]
\begin{center}

    \subfloat[$d= 5\, \mathrm{mm}$]{
    \includegraphics[width=75mm]{figures/datavseta5.pdf}
    }    
    \subfloat[$d= 10\, \mathrm{mm}$]{
    \includegraphics[width=75mm]{figures/datavseta10.pdf}
    }
     \caption{データ伝送速度-電力効率特性の測定結果}\label{datavseta} 
  \end{center}
\end{figure}


\subsubsection{文字列データ伝送実験}
送受信側にそれぞれPCを接続し,文字列を伝送する実験を行った.データの送受信には,PCのUSB端子と,Windows用のシリアル通信ソフトウェアTera Termを用いた.このときの実験系を図\ref{communication}に示す.なお,受信側回路の電源については,安定動作を図るために,PCのUSB端子から供給した.実験手順を以下に示す.

\begin{enumerate} \setlength{\itemsep}{-0.2cm}
  \item コイル間距離$d =10 \, \mathrm{mm}$, ハーフブリッジの入力電圧$V_{in}=5 \, \mathrm{V}$とした.
  \item 送電基板のデータ入力端子に,BER測定器からビットレート300 bpsの文字列データ信号を送信した.
  \item 受信側FPGAで復調したデータを,USB端子を経由してPC上のシリアルモニタに表示させ,文字列が正しく伝送されることを確認した.また,その際の直流出力電圧$V_{out}$ならびに入力電流$I_{in}$を測定し,電力伝送効率を求めた.
  \item ビットレートを徐々に上げながら,通信が不可能となるまで測定を行った.
  \end{enumerate}
送受信された文字列を目視で照らし合わせ確認したところ,ビットレート57.6 kbpsまでは誤り無く通信可能であった.このときの電力伝送効率は32.4\%であった.
\begin{figure}[h]
\begin{center}

\includegraphics[width=160mm]{figures/communication.pdf}
  \caption{文字列データ伝送実験の測定系}
  \label{communication}

  \end{center}
\end{figure}


\subsubsection{BERの測定}
データの伝送特性をより定量的に把握するため,符号誤り率(Bit Error Rate : BER)の測定を行った.このときの実験系を図\ref{ber}に示す.実験手順は次のとおりである.
\begin{enumerate} %\setlength{\itemsep}{-0.2cm}
  \item コイル間距離$d =10 \, \mathrm{mm}$, ハーフブリッジの入力電圧$V_{in}=5 \, \mathrm{V}$とした.
  \item 変調基板のデータ入力端子に,BER測定器からビットレート57.6 kbps の擬似ランダム符号を送信した.擬似ランダム符号の繰り返し周期は$2^{15}-1$(PN15系列)とした.
  \item $10^7$ビットの送信が完了した後,受信側のBER測定器に表示されたBERを記録した.
  \item ビットレートを3 kbpsごと66.6 kbpsまで上げながら,同様の測定を行った.測定は3回行い,その平均値を結果とした.
  \item コイル間距離$d =5 \, \mathrm{mm}$,$15 \, \mathrm{mm}$として,同様の測定を行った.ただし,ここでは測定を1回のみとした.
  \end{enumerate}

\begin{figure}[h]
\begin{center}

\includegraphics[width=160mm]{figures/ber.pdf}
  \caption{BERの測定系}
  \label{ber}

  \end{center}
\end{figure}

測定結果を図\ref{bergraph}に示す.同図の縦軸が対数軸であることに留意すれば,$d=1,1.5\, \mathrm{cm}$のとき,BERはビットレートに対して指数関数的に増大している.一方,$d=0.5\, \mathrm{cm}$では他の2つと異なり,ビットレートが低い場合でもBERが十分に小さくならない特性が認められる.

\begin{figure}[h]
\begin{center}

\includegraphics[width=120mm]{figures/bergraph.pdf}
  \caption{BERの測定結果}
  \label{bergraph}

  \end{center}
\end{figure}

            

%参考文献
%\bibliography{bib_for_thesis}
%\bibliographystyle{sieicej}
\begin{thebibliography}{9}
  \bibitem{Reserver}田中剛平, 中根了昌, 廣瀬明, 「リザバーコンピューティング -時系列パターン認識のための高速機械学習の理論とハードウェア-」, 森北出版, 2021-5
  \bibitem{Sunada23}T.Yamaguchi, K.Arai, T.Niiyama , A.Uchida, and S.Sunada, \"Ultrafast single-channel machine vision based on neuro-inspired photonic computing\", Communications Physics 6, 250, 2023
  %\bibitem{}
\end{thebibliography}

\chapter*{謝辞}
\addcontentsline{toc}{chapter}{謝辞}
本研究を遂行するにあたり,大変手厚く御指導頂いた本学電気電子生命学科和田和千准教授に深く感謝する.併せて,本論文の執筆にあたり有益なる御助言を頂いた同学科通信伝送グループの井家上哲史教授,関根かをり教授,中村守里也准教授に深く感謝する.また,日頃の研究において,議論を通じて多くの御助言を頂いた波動信号処理回路研究室諸氏に厚く感謝する.

\makesignature


%\appendix
%\chapter{実装回路図ならびにPCBレイアウト図}
実装した送受信回路の回路図ならびにPCBレイアウト図を次頁以降に添付する.

\begin{landscape}
\begin{figure}[p]
\begin{center}

\includegraphics[width=220mm]{figures/wpf13_circuit1.pdf}
  \caption{送信側回路図(1/2)}

  \end{center}
\end{figure}
\end{landscape}


\begin{landscape}
\begin{figure}[p]
\begin{center}

\includegraphics[width=220mm]{figures/wpf13_circuit2.pdf}
  \caption{送信側回路図(2/2)}

  \end{center}
\end{figure}
\end{landscape}


\begin{landscape}
\begin{figure}[p]
\begin{center}

\includegraphics[width=220mm]{figures/wpf_receiver7_circuit.pdf}
  \caption{受信側回路図}

  \end{center}
\end{figure}
\end{landscape}


\begin{figure}[p]

	\begin{center}
    \subfloat[表面]{
    \includegraphics[width=150mm]{figures/wpf13_board_002.pdf}
    }
    \\ 
    \subfloat[内層1面]{
    \includegraphics[width=150mm]{figures/wpf13_board_003.pdf}
    }
    \\ 
  \end{center}
\end{figure}

\begin{figure}[p]
	\begin{center}
    
    \setcounter{subfigure}{2}
    
    \subfloat[内層2面]{
    \includegraphics[width=150mm]{figures/wpf13_board_004.pdf}
    }
    \\ 
   
    \subfloat[裏面]{
    \includegraphics[width=150mm]{figures/wpf13_board_005.pdf}
    }

  \caption{送信側PCBレイアウト図}

    \end{center}
\end{figure}
  
  \begin{figure}[p]

	\begin{center}
    \subfloat[表面]{
    \includegraphics[width=150mm]{figures/wpf_receiver7_board_002.pdf}
    }
    \\ 
    \vspace{5mm}
    \subfloat[裏面]{
    \includegraphics[width=150mm]{figures/wpf_receiver7_board_003.pdf}
    }
    \\ 
    \caption{受信側PCBレイアウト図}
  \end{center}
\end{figure}

\chapter{FSK復調器のVerilog-HDLソースコード}
FPGAを用いたFSK復調器のVerilog-HDLソースコードならびに論理シミュレーション用テストベンチを以下に示す.

\begin{lstlisting}[caption=FSK復調器]

module counter7_for_thesis(
    input clk,
    input serial_datain, 
    input sw0,
    input sw1,
    input sw2,
    input sw3,
    input btn0,
    input extfsksig, //b10
    
    output serial_dataout,
    output led0,
    output b1,
    output b2,
    output b3,
    output b4,
    output b7,
    output b8,
    output b9
        );  


parameter low_freq=598; /*kHzで指定*/  
parameter high_freq=979;
parameter integer cnt_fspace=(1/(low_freq*2*0.000008));
parameter integer cnt_fmark=(1/(high_freq*2*0.000008));  
parameter integer threshold=167;

reg [19:0] cntdata=20'd0;
reg [31:0] cntfspace=32'd0;
reg [31:0] cntfmark=32'd0;
reg [31:0] demodcnt_1p=32'd0;
reg [31:0] demodcnt_2p=32'd0;
reg [31:0] demodcnt_p=32'd0;
reg [31:0] difference_p=32'd0;
reg [31:0] bps=32'd0;
reg [31:0] rstpls_p=32'b0;
reg [63:0] resetcounter=64'd0; 

reg fmark=1'b0; /* high freq*/
reg fspace=1'b0; /*low freq*/
reg fsksig=1'b0;
reg fdata=1'b0;
reg datachange=1'b0;
reg demodout=1'b0;
reg cntrst_p=1'b0;    
reg resetflag=1'b0;
reg sigselect=1'b0;
reg recoverflag=1'b0;

initial @(negedge demodout) recoverflag<=1'b1;

always @(posedge clk)begin


/* set rate of pseudo data */
/////////////////////////////////////////////////////
       
         if(sw3==0 & sw2==0 & sw1==0& sw0==0) bps<='d300;
    else if(sw3==0 & sw2==0 & sw1==0& sw0==1) bps<='d600;
    else if(sw3==0 & sw2==0 & sw1==1& sw0==0) bps<='d1200;
    else if(sw3==0 & sw2==0 & sw1==1& sw0==1) bps<='d2400;
    else if(sw3==0 & sw2==1 & sw1==0& sw0==0) bps<='d4800;
    else if(sw3==0 & sw2==1 & sw1==0& sw0==1) bps<='d9600;
    else if(sw3==0 & sw2==1 & sw1==1& sw0==0) bps<='d14400;
    else if(sw3==0 & sw2==1 & sw1==1& sw0==1) bps<='d19200;
    else if(sw3==1 & sw2==0 & sw1==0& sw0==0) bps<='d38400;
    else if(sw3==1 & sw2==0 & sw1==0& sw0==1) bps<='d57600;
    else if(sw3==1 & sw2==0 & sw1==1& sw0==0) bps<='d115200;
    else if(sw3==1 & sw2==0 & sw1==1& sw0==1) bps<='d230400;
    else if(sw3==1 & sw2==1 & sw1==0& sw0==0) bps<='d460800;
    else if(sw3==1 & sw2==1 & sw1==0& sw0==1) bps<='d921600;
    else bps<='d115200;
    
/////////////////////////////////////////////////////


/*genarate pseudo FSK signal*/
/////////////////////////////////////////////////////
    if(cntfspace == cnt_fspace) begin
        cntfspace<=32'd0; // reset cnt //
        fspace<=~fspace;
 end
        
else begin
    cntfspace<=cntfspace+32'd1;
 end
 
////////////////////////////////////////////
 
     if(cntfmark == cnt_fmark) begin
        cntfmark<=32'd0; // reset cnt //
        fmark<=~fmark;
 end
        
else begin
    cntfmark<=cntfmark+32'd1;
 end
 
////////////////////////////////////////////

      if(cntdata == 'd125000000/bps) begin
        cntdata<=20'd0; // reset cnt //
       fdata<=~fdata;
 end
        
else begin
    cntdata<=cntdata+20'd1;
 end
 
///////////////////////////////////////////////////// 
  
  
/*choose internal(pseudo) or external FSK signal*/
/////////////////////////////////////////////////////
 
 if(btn0==1) sigselect<=~sigselect;
 
 if(sigselect==0) begin
 
    if(fdata==0)   begin
    fsksig<=fspace;
    end
     else begin
    fsksig<=fmark;
    end
 end
 
 if(sigselect==1) fsksig<=extfsksig;
 
/////////////////////////////////////////////////////


/*generate counter-reset pulse*/
/////////////////////////////////////////////////////

if(fsksig==1'b1) begin  
    rstpls_p<=rstpls_p+1'b1;
        if(rstpls_p<1'b1) cntrst_p<=1'b1;
        else if (rstpls_p==1'b1) begin
        rstpls_p<=1'b1;
        cntrst_p<=1'b0;
        end

end

if(fsksig==1'b0) rstpls_p<=1'b0;
/////////////////////////////////////////////////////


/*set counter*/
/////////////////////////////////////////////////////
 
  if(fsksig==1'b0 && cntrst_p==1'b0)      demodcnt_2p<=demodcnt_2p+32'd1;
 else if (fsksig==1'b1 && cntrst_p==1'b0)    demodcnt_1p<=demodcnt_1p+32'd1;
 else begin
    demodcnt_1p<=32'd0;
    demodcnt_2p<=32'd0;
 end
 
 demodcnt_p<=demodcnt_1p+demodcnt_2p;
 
///////////////////////////////////////////////////// 
 
 
/*set normally HIGH*/
/////////////////////////////////////////////////////
if(cntrst_p==1'b1) begin
    if(demodcnt_1p>demodcnt_2p) difference_p<=demodcnt_1p-demodcnt_2p;
    if(demodcnt_2p>demodcnt_1p) difference_p<=demodcnt_2p-demodcnt_1p;
    if(demodcnt_1p==demodcnt_2p) difference_p<=0;
end

if(difference_p>'d10) datachange<=1'b1;
else datachange<=1'b0;

////////////////////////////////////////////

if(datachange==0)begin                      // set normally high
    resetcounter<=resetcounter+32'd1;
    
    if(resetcounter>'d125000000) resetflag<=1'b1; else resetflag<=1'b0;
end

if(datachange==1) resetcounter<=32'd0;
    
/////////////////////////////////////////////////////


/* data detection */
/////////////////////////////////////////////////////

if(cntrst_p==1'b1) begin
        if(demodcnt_p>threshold) demodout<=1'b0;
        if(demodcnt_p<threshold) demodout<=1'b1;
    end 
    
   else demodout<=demodout;
   
///////////////////////////////////////////////////// 
 end
 
assign b1=fdata;
assign b2=fmark;
assign b3=fspace;
assign b4=fsksig;
assign b7=demodout;
assign b8=cntrst_p;
assign b9=datachange;
assign led0=sigselect;
assign serial_dataout=demodout;
endmodule

\end{lstlisting}


\begin{lstlisting}[caption=テストベンチ]

`timescale 1ns / 1ps

module counter7_tb;
    
    reg clk;
    reg rnd;
    
    wire sw0=0;
    wire sw1=0;
    wire sw2=0;
    wire sw3=1;
    wire btn0=0;
    wire extfsksig=0;
           
    wire serial_dataout;
    wire led0;
    wire b1;
    wire b2; 
    wire b3; 
    wire b4;
    wire b7;
    wire b8;
    wire b9;
   
    initial clk<=0;
  
    always  #4 clk=~clk;
    
    counter7_for_thesis uut(
    .clk(clk), 
    .serial_datain(rnd),
    
    .sw0(sw0),
    .sw1(sw1),
    .sw2(sw2),
    .sw3(sw3),
    .btn0(btn0),
    .extfsksig(extfsksig),
    
    .serial_dataout(serial_dataout),
    .led0(led0),
    .b1(b1), 
    .b2(b2), 
    .b3(b3),
    .b4(b4),
    .b7(b7),
    .b8(b8),
    .b9(b9));
endmodule
\end{lstlisting}


\end{document}
