\chapter{結論}

    本研究では既存のギルバート乗算回路を並列に用いることで考えられる問題点である,「信号振幅の減少によるS/N劣化」を防ぐ回路構成についての提案を行った.提案回路についてギルバート乗算回路同様小信号解析により入
    力電圧と制御電圧の両方に比例した電圧出力得られることを示した.また,ギルバート乗算回路に比してしきい電圧分であるおよそ$0.45\;\mathrm{V}$程度,振幅の範囲を拡大することができる可能性を示した.\par

    第2章ではギルバート乗算回路について定性的な動作について述べた.その後,ゲート接地差動対におけるトランスコンダクタンスの増減とゲート電圧の関係式を二乗則より導出した.そしてその結果を用いて,ギルバート乗算回路全体の小信号解析を行い,具体的な出力電圧の理論式を導いた.\par

    第3章では既存のギルバート乗算回路より出力振幅を拡大する回路の案としてPMOSFETを用いた折り返しカスコード型の回路構成を提示し,折り返しカスコードの問題点を指摘した.そしてその問題点を改善すべく,カレントミラーを組み合わせた折り返し型乗算回路を提案しその回路動作について説明した.さらに小信号解析においてカレントミラーを組み合わせた折り返し型乗算回路がギルバート乗算回路同様に入力電圧と制御電圧の両方に比例した電圧出力を得られることを導出した.また,ギルバート乗算回路と提案回路について$\mathrm{ROHM\;0.18\;\mu m}$標準CMOSプロセスにおける素子値の設計とその時の直流・交流解析結果を示し,シミュレーション上で出力範囲を拡大できることを確認した.\par

    今後の主な研究課題は高速化と並列化である.本研究ではフォトニックリザバの後処理部に応用することを念頭に置いているため,フォトニックリザバの動作速度よりも低速で動作することは避けたく,そちらの性能を生かすためにも高速化は必須である.さらに,出力端子も1つではないので多入力化に対応させる必要がある.高速化に関してはRFで利用されているギルバート乗算回路を参考にさらに微細なプロセスで追及していこうと考えている.