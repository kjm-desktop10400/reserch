\documentclass{jsarticle}
\usepackage[dvipdfmx]{graphicx}
\usepackage{comment}
\title{\vspace{-3cm}磁界共鳴方式ワイヤレス電力伝送の周波数特性に着目した \\ FSK変調方式電力・データ同時伝送システムの設計(46p)}
\date{\empty}

\begin{document}
\maketitle 
\vspace{-2cm}

\section{序論(4p)}
\subsection{研究背景(3p)}
無線電力伝送,あるいは電力とデータの同時伝送について,その現状あるいは課題などを述べる.特に,医療用の埋め込み機器等の分野において,電力とデータを同時に伝送するシステムの研究開発が進められていることを,適宜参考文献を提示して述べる.

\subsection{研究目的(1p)}
電力とデータを同時に伝送するシステムの方式は種々提案されているものの,MR-WPT回路の周波数特性の双峰性に着目したものは少ないことを述べる.本研究では,双峰性に着目することにより,従前の方式と比較して大きな電力とデジタルデータとを同時に伝送するシステムを設計することを目的とする.

\section{MR-WPT回路の原理(8p)}
\subsection{MR-WPT回路の特性(5p)}
MR-WPTの原理回路図を示し,その特性を数式ならびに数値計算から得られた各種グラフを用いて説明する.結合係数$k$の変動によらず大電力を送電するために,ZRFの追従が有用であることを述べる.

\subsection{ZRFの解析(3p)}
前項におけるMR-WPTの回路解析結果を用いて,ZRFの解析式を導出する.また,ZRFが複数存在する負荷抵抗ならびに結合係数$k$の条件を導出する.

\section{PLLを用いたZRFの自動追従回路(12p)}

\subsection{原理(5p)}
\subsubsection{PLLの動作(2p)}
PLLの原理図を示し,PLLがZRFの追従に有用であることを述べる.

\subsubsection{PLLが所望のZRFでロックすることの証明(3p)}
PLLが,3つあるZRFのうち所望の2つのみでロックすることを,フルビッツの安定判別法により示す.

\subsection{回路構成(4p)}
\subsubsection{全体構成(2p)}
ハーフブリッジ等を含めた実際の回路構成を示し,その動作について述べる.

\subsubsection{デッドタイム生成回路(1p)}
デッドタイム生成回路の回路図を示し,その動作について述べる.

\subsubsection{波形整形・遅延補正回路(1p)}
波形整形・遅延補正回路の回路図を示し,その動作について述べる.

\subsection{シミュレーション(3p)}
LTspiceにおけるシミュレーションの回路図ならびにその結果を示し,前述した回路構成で所望の動作が実現できることを述べる.

\begin{comment}
\subsection{実装ならびに実験(6p)}
\subsubsection{実装回路(1p)}
実装した回路の写真を示し,実装時に留意した点を述べる.

\subsubsection{実験構成・実験方法(2p)}
実験系の構成図を示し,実験方法を述べる.

\subsubsection{実験結果(3p)}
実験結果とその考察を述べる.
\end{comment}

\section{FSK変調方式による電力・データ同時伝送システム(24p)}
\subsection{原理ならびにシステムの全体構成(3p)}
2つのPLLを用意し,それらの出力を切り替えることでFSK変調器が構成できることを述べる.全体構成図を併せて示す.

\subsection{FSK変調回路(3p)}
\subsubsection{全体構成(1.5p)}
FSK変調回路の構成図を示し,その概要を述べる.
\subsubsection{電圧制限回路(0.5p)}
VCOの発振周波数を制限するための電圧制限回路について述べる.
\subsubsection{Lock Detector回路(0.5p)}
ロック状態を検出し,LPFの時定数を切り替えるLock Detector回路について述べる.
\subsubsection{Timing Controller回路(0.5p)}
データの遷移時にPLLが誤動作することを防止する回路Timing Controller回路について述べる.

\subsection{シミュレーション(4p)}
システム全体のシミュレーション回路図ならびにシミュレーション結果を示し,上述した変調回路の動作を確認する.


\subsection{FPGAを用いたFSK復調器(6p)}
\subsubsection{動作原理(3p)}
FPGAによる復調器の原理について述べる.

\subsubsection{FPGA単体でのFSK変復調実験(3p)}
FPGA内部に変復調回路を構成することにより,最大レート115.2kbpsで文字列データを送受信できることを,ロジックアナライザで測定した波形を示して述べる.


\subsection{実装ならびに実験(8p)}
\subsubsection{実装回路(1p)}
実装した回路の写真を示し,その説明を述べる.
\subsubsection{コイル間距離-結合係数特性の測定(1p)}
実験方法ならびにその結果を述べる.

\subsubsection{ZRFの手動追従実験(1p)}

\subsubsection{ZRFの自動追従実験(1p)}

\subsubsection{データ伝送時の出力スペクトルの測定(1p)}

\subsubsection{データ伝送速度-電力効率特性の測定(1p)}

\subsubsection{文字列データ伝送実験(1p)}

\subsubsection{考察(1p)}
実験結果をまとめ,考察を述べる.

\section{結論(2p)}
MR-WPT回路の周波数特性に着目した電力とデータの同時伝送システムを設計し,電力伝送効率32.4\%,データレート57.6kbps(2019年12月現在)で電力とデータの同時伝送が実現できたことを述べる.今後の課題として,データ伝送速度の高速化,回路の集積化,電力伝送効率の向上などを挙げる.

\section*{参考文献}
\section*{研究業績}
\section*{謝辞}



\appendix
\section{}実装回路の全体回路図ならびにPCBレイアウト図
\section{}FSK復調器のVerilog-HDLソースコード
\end{document}
