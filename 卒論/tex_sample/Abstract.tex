\chapter*{概要}
磁界共鳴方式無線電力伝送(Magnetic Resonance Wireless Power Transfer : MR-WPT)回路における周波数特性の双峰性に着目し,FSK変調方式を用いて電力とデータを同時に伝送するシステムを提案する.まず,MR-WPT回路の解析を行いその周波数特性を明らかにし,伝送コイル間の結合状態が何らかの要因により変動する状況下で大電力を伝送するためには,適切な周波数制御回路が必要であることを述べる.次に,周波数制御回路として位相同期ループによるフィードバックを用いた回路を示し,同回路の伝達関数を導出するとともに,その動作をシミュレーションにより確認する.さらに,周波数制御回路を応用し,MR-WPT回路の駆動周波数をバイナリデータに対応させて切り替えることにより,電力とデータを同時に伝送するシステムを提案する.提案システムの有効性について,シミュレーションならびにプリント基板上に実装した回路を測定することにより確認する.結果として,伝送コイル間距離$10 \, \mathrm{mm} \, (\mbox{結合係数}k \simeq 0.36$)の条件下において,電力伝送効率$32.4 \%$,ビットレート57.6 kbpsでの電力とデータの同時伝送を達成する.
