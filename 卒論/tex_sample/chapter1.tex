\chapter{序論}
近年,スマートフォン等のモバイル機器,医療用埋め込み機器,あるいは電気自動車等への給電を主たる目的として,無線電力伝送の研究が盛んに行われている.電力伝送の方式は,マイクロ波の遠方界を利用するもの\cite{Saito2013},レーザー光\cite{Duncan2016}を用いるものなど種々提案されているが,2007年のMITの研究\cite{Kurs2007}に端を発する磁界共鳴方式は,高効率かつ大電力の伝送が可能であることから広く研究されている\cite{Anyapo2017,Fu2016,Iordache2015,Tsuchida2018a,Tsuchida2018,Zhao2017,Wenxian2014,Imano2014,Li2015,Narusue2015,Gati2015,Jiwariyavej2015,Zhen2019,Awai2012,Hosotani2012,Yang2017,Chaidee2017,Cenk2017,Imura2017,Fujita2019a}.この方式は,送受電コイル間に誘起される交番磁界により電力を伝送するものであり,基本的には変成器と同様の原理であるが,送受電コイルにキャパシタを接続して共振現象を利用するという点で従来の変成器と異なる.同方式による伝送回路では,伝送コイル間の結合係数$k$や負荷抵抗の値によっては,出力電力が極大値となる駆動周波数,すなわち極大電力周波数が2つ存在し,これはFrquency Splitting PhenomenaあるいはBifurcation Phenomenaとして広く知られている現象である\cite{Wang2004,Hoeher2019,Niu2013}.\par
一方,無線電力伝送の応用分野として,電力とデータの同時伝送(Simultaneous Wireless Information and Power Transfer : SWIPT)がある\cite{Yakovlev2012,Krikidis2014,Kim2019,Wu2015,Ishii2018,Nguyen2015,Hoeher2019,LotfiNavaii2018,Fujita2019b}.無線電力伝送は,電磁波を利用して電力を送るものであるから,それに適当な変調を加えてデータを同時に伝送するという考えはごく自然であるといえる.従来は電力の伝送のみに用いられていた電磁波をデータの伝送にも活用することは,周波数資源の有効利用,あるいは機器の小型化,省電力化といった観点から有用であると考えられる.\par
磁界共鳴方式で電力とデータの同時伝送を実現する先行研究例は多くなく,こと先述したFrquency Spilitting Phenomenaに着目したものは少ない.文献\cite{Hoeher2019}は,同現象に着目し,2つの極大電力周波数にバイナリデータを割り当てる,すなわちFSK変調を行うことにより電力とデータの同時伝送を実現するものであるが,伝送コイル間の結合あるいは負荷の変動については考慮されていない.これらが動的に変動するような用途を想定した場合は,適切な制御回路を付加する等の方法により,変動に起因する影響を補償する必要がある.\par 
本研究では,磁界共鳴方式におけるFrquency Splitting Phenomenaを活用し,FSK変調による電力とデータの同時伝送を実現する手法について検討するとともに,位相同期ループ(Phase-Locked Loop : PLL)を用いて伝送コイル間の結合あるいは負荷の動的な変動を補償する回路を提案する.本論の構成は次のとおりである.まず,2章において磁界共鳴方式無線電力伝送回路の回路図を示し,回路方程式を解くことによりその特性を定量的に明らかにする.次に3章において,PLLを用いた回路を提案してその伝達関数モデルを示すとともに,シミュレーションにより所望の動作が実現できることを述べる.次に,4章においてFSK変調による電力とデータの同時伝送手法について述べ,シミュレーションによる動作確認を行うとともに,プリント基板上に実装した回路の実験結果を示し,提案手法ならびに実装回路の特性を明らかにする.最後に,5章で結論ならびに今後の展望を述べる.
