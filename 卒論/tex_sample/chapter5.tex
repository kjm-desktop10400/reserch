\chapter{結論}
本研究では,磁界共鳴方式無線電力伝送(Magnetic Resonance Wireless Power Transfer : MR-WPT)回路の周波数特性の双峰性に着目し,FSK変調方式により電力とデータを同時に伝送するシステムを設計した.設計システムをプリント基板上に実装して動作実験を行った結果,伝送コイル間距離$10 \, \mathrm{mm} \, (\mbox{結合係数}k \simeq 0.36$)の条件下において,電力効率$32.4 \%$,ビットレート57.6 kbpsで電力とデータの同時伝送が行えることを確認した.\par 
第2章においては,MR-WPT回路の等価回路から,回路方程式を用いてその動作を解析した.同回路において,出力電力が極大値をとる周波数が2つ存在する現象(Frequency Splitting Phenomena)について述べ,それらの周波数はコイル間結合係数$k$や負荷抵抗の値により変動することを示した.大きな電力を伝送するためには極大電力周波数を追従する回路が必要であり,その動作の実現のために零リアクタンス周波数(Zero Reactance Frequency : ZRF)が重要であることを述べ,ZRFの解析式を導出した.\par 
第3章では,MR-WPT回路をZRFで駆動するために,位相同期ループ(Phase-Locked Loop : PLL)を用いた回路を提案した.MR-WPT回路中の電圧ならびに電流の位相をPLLで比較するフィードバックループを構成することにより,所望の動作が実現できることを述べた.PLLを含む回路全体の伝達関数を導出することにより,その動作を定量的に把握することを試みるとともに,PLL内のLPFのトポロジーについて比較検討を行った.また,LTspiceを用いたシミュレーションにより,提案回路が正しく動作すること確認した.\par 
第4章では,前章で述べたPLLによる回路を応用し,MR-WPT回路を駆動する周波数をバイナリデータにより変化させる,すなわちFSK変調を行うことにより,電力とデータを同時に伝送する回路を提案した.また,FPGAを用いたFSK復調器を実装し,その動作原理について述べるとともに,ロジックアナライザを用いて動作確認を行った.提案回路は,設計したプリント基板上に実装し,全7種類の実験を通して所望の動作が実現できることを示すとともに,種々の特性を明らかにした.実験により明らかになった重要な特性は,電力伝送効率はビットレートにほぼ依存しないことと,BERはコイル間の結合係数に依存するある値を境に急激に悪化することである.\par 
今後の主な研究課題は,電力伝送効率ならびにビットレートの向上である.電力伝送効率の向上を達成するためには,システムの低消費電力化,あるいは受電側整流回路の高効率化が必須である.例えば,現行の回路は市販のICを組み合わせて実装しているが,これらを包含した集積回路を新規に設計することにより,システム全体の電力消費を低減することが考えられる.さらに,本研究で実装した整流回路は,ショットキーダイオードを4つ用いたブリッジ整流回路であるが,これを半導体スイッチを用いた同期整流回路に置換することにより,ダイオードの順方向電圧降下$V_F$に起因する整流損を低減することができると考えられる.また,ビットレートの向上を実現するためには,実験で明らかになったBER特性をより詳細に分析して,BERが急激に悪化する原因を明らかにするとともに,必要に応じて送受信回路の構成やFPGAを用いたFSK復調器のコードを改良する必要がある.
