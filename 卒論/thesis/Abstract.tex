\chapter*{概要}
フォトニックリザバの学習・計算の際に必要となるアナログ的な積和演算を可能とする乗算回路の構成を提案する.KCLを利用することで,多数の信号の和を電流的に実現することができるが,信号線を共有するため和の出力振幅は単体の乗算回路と共有される.即ち,積和演算回路の出力振幅は乗算回路単体の出力振幅と等しくなり,和をとる信号の数が多くなればなるほど乗算回路一つあたりの入力範囲が限られてしまうため信号対雑音比(S/N 比)の劣化が懸念される.そこで,従来のギルバート乗算回路を応用した乗算回路を提案しギルバート乗算回路同様,アナログ信号の乗算ができることを,小信号解析を用いて確認した.また,乗算器が動作する条件として,MOSFETが遮断しない条件を使い出力範囲を示した.そして,今後集積化を行い出力範囲拡大ができていることを確認するための素子値を設計し,パッケージでの測定を踏まえたシミュレーションを行った.
