\chapter*{概要}
フォトニックリザバの学習・計算の際に必要となるアナログ的な積和演算を可能とする乗算回路の構成を提案する.多数の信号に乗算回路で重み付けした後で和を得るために,複数の乗算出力信号をそれぞれ電流で表し,それら出力端子を結線しさえすればKCLに基づき実現することができる.しかしながら信号線を共有するため和の出力振幅は単体の乗算回路と共有される.即ち,積和演算回路の出力振幅は乗算回路単体の出力振幅と等しくなり,和をとる信号の数が多くなればなるほど乗算回路一つあたりの入力範囲が限られてしまうため信号対雑音比(S/N 比)の劣化が懸念される.そこで,従来のギルバート乗算回路を2つの増幅回路の縦続接続と考え,一方を折り返した構造により動作範囲を拡大した回路を提案ししている.そしてアナログ信号の乗算ができることを,小信号解析を用いて確認している.また,乗算器が動作する条件を解析し,MOSFETが遮断しない条件を使い出力範囲を示しす.そして,設計例のシミュレーションにより出力範囲拡大ができることを確認している.パッケージでの測定を踏まえたシミュレーションを行った.
