\documentclass[twocolumn]{jsarticle}
\usepackage[dvipdfmx]{graphicx}
\usepackage{kenzemi}
\usepackage{amsmath}
\usepackage{amssymb}
\usepackage{mathtools}
\usepackage{float}

\setcounter{topnumber}{100}
\setcounter{bottomnumber}{100}
\setcounter{totalnumber}{100}
\renewcommand{\textfraction}{0.0}
\renewcommand{\topfraction}{1.0}
\renewcommand{\bottomfraction}{1.0}

\renewcommand{\dbltopfraction}{.7}
\renewcommand{\dblfloatpagefraction}{.5}

\mathtoolsset{showonlyrefs = true}

\begin{document}

\title{ギルバート乗算回路における周波数特性改善方針の検討}
\author{小島 光}
\date{2023年10月31日}
\abstract{以前まで検討していた折り返し型のギルバート乗算回路の問題点である周波数特性劣化の原因がPMOSであると仮定し、これを使わない新たなトポロジについて検討した。}
\keyword{ギルバート乗算回路,小信号解析,周波数特性}
\maketitle

\section{はじめに}
    フォトニックリザバコンピューティングの学習には高速な積和演算が必要であるが現状、要件を満たすような光を用いた積和演算行えていない。そこでリザバ層の出力を電気に変換しアナログ積和演算を行うことになった。ここで、リザバ層からは7つの出力があり、それぞれに任意の重みをかけ足し合わせる。これを電気で行うには図\refeq{fig:gilbert_conv}のような各ギルバートセルで積算を行い、電流をまとめて和をとる。即ち信号振幅を足し合わせるため7つの信号振幅の和が必要な振幅になる。つまり各セルの出力振幅は全体の$1/7$が最大となり、S/N比が小さくなってしまうことが問題として考えられる。この問題を改善すべく前回まではPMOSを使用する折り返し型のギルバート乗算回路(図\refeq{fig:gilbert_folded_NtoP})について検討してきたが信号振幅を改善できても周波数特性が劣化してしまうことが分かった。
    \begin{figure}[h]
        \begin{center}
            \includegraphics*[width=80mm]{figures/gilbert.png}
            \caption{従来型のギルバートセル}
            \label{fig:gilbert_conv}
        \end{center}
    \end{figure}
    \begin{figure}[H]
            \includegraphics*[width=80mm]{figures/folded_gilbert.png}
            \caption{PMOSを使用した折り返し型ギルバートセル}
            \label{fig:gilbert_folded_NtoP}
    \end{figure}

\section{三端子間に同時に寄生容量がついた場合の小信号等価解析}
    周波数特性劣化の原因としてPMOSについた寄生容量が考えれる。そこで三端子間に寄生容量としてキャパシタを挿入した小信号等価回路を解析することにより寄生容量の影響を検討する。図\refeq{label:paracitic_eq}に計算の際に考えた小信号等価回路を示す。
    \begin{figure}[h]
        \begin{center}
            \includegraphics*[width=80mm]{figures/ParasiticCapacitoresEquivalent.png}
            \caption{寄生容量を考慮した小信号等価回路}
            \label{label:paracitic_eq}
        \end{center}
    \end{figure}
    各部に流れる電流は各接点の電位などを用いて  
    \begin{gather*}
        i_{ul}=g_{mn}v_{in}-g_{dn}v_{pS1}   \\
        i_{lSG}=i_{rSG}=j\omega C_{GS}v_{pS1}       \\
        i_{lGD}=j\omega C_{GD}v_{out+}      \\
        i_{rGD}=j\omega C_{GD}v_{out-}      \\
        \therefore i_{lGD}+i_{rGD}=0\;\;\;(\because v_{out+}=-v_{out-})     \\
    \end{gather*}
    \begin{multline*}
        i_{ll} = (g_{mp}+\Delta g_{m})v_{pS1}\\+(g_{dp}+j\omega C_{SD})(v_{pS1}-v_{out+})-i_{lGD}
    \end{multline*}
    \begin{multline*}
        i_{ll} = (g_{mp}-\Delta g_{m})v_{pS1}\\+(g_{dp}+j\omega C_{SD})(v_{pS1}-v_{out-})-i_{rGD}    
    \end{multline*}
    KCLより
    \begin{align*}
        i_{ul} &= i_{ll}+i_{lGD}+i_{lSG}+i_{lr}+i_{rGD}+i_{rSG}    \\
        &= i_{ll}+i_{lr}+2i_{lSG}   \\
        g_{mn}v_{in}-g_{dn}v_{pS1} &= 2\left\{ g_{mp}v_{pS1}+g_{dp}v_{pS1}+ j\omega( C_{SD}+C_{SG} ) v_{pS1} \right\}   \\
        v_{pS1} &= \frac{g_{mn}}{ 2g_{mp}+g_{dn}+2g_{dp}+j2\omega (C_{SD}+C_{SG}) }v_{in}
    \end{align*}
    ここで、$g_{mp}\ll g_{dp},g_{dn}$を仮定すると
    \begin{align}
        v_{pS1} \approx \frac{g_{mn}}{ 2\left\{ g_{mp}+j\omega(C_{SD}+C_{SG}) \right\} }    \label{eq:vps1}
    \end{align}
    となる。ここで、負荷抵抗に流れる電流は差動半回路の性質より$i_{lr}=-i_{rl}$となるので,KCLより
    \begin{align*}
        i_{out+}=i_{ll}+i_{rl}=i_{ll}-i_{lr}
    \end{align*}
    である。また、差動半回路であるので対応する電流・電圧は符号が反対で絶対値が等しいので
    \begin{align*}
        v_{out} &= v_{out+}-v_{out-}    \\
        &=R_{L}(i_{out+}-i_{out-})      \\
        &= 2R_{L}i_{out+}
    \end{align*}
    \eqref{eq:vps1}を用いると
    \begin{align*}
        v_{out} &= 2R_{L} \left[ 2\Delta g_{m}v_{pS1}- \left\{ g_{dp} + j\omega (C_{SD} + C_{GD}) \right\} \right]    \\
        &= \frac{4R_{L}}{ 1+2R_{L}\left\{ g_{dp} + j\omega(C_{SD}+C_{GD}) \right\} } \Delta g_{m}v_{pS1}
    \end{align*}
    ここで、前回のゼミ発表より$\Delta g_{m}$は定数$K$を用いて
    \begin{align*}
        \Delta g_{m} = 2KV_{CTRL}
    \end{align*}
    だったので出力は
    \begin{multline}
        v_{out} \\
        = \frac{4KR_{L}g_{mn}\cdot V_{CTRL}\cdot v_{in}}{ \left\{ 1+j2R_{L}\omega(C_{SD}+C_{GD}) \right\}\left\{ g_{mp}+j\omega(C_{SD}+C_{SG}) \right\} }    \label{eq:vout}
    \end{multline}
    と求められた。

    折り返し型のギルバート乗算回路でシミュレーションを行い、.op解析からコントロール電圧毎のパラメータを抽出した。この時のシミュレーション条件を表\refeq{table:state_folded_ac}に示す。また、この時のシミュレーション波形と.op解析で得られたパラメータを代入した理論式のグラフを図\refeq{fig:vout_theoretical}、\refeq{fig:vout_sim}に示す。
    \begin{table}[h]
        \caption{折り返し型ギルバート乗算回路の素子値}
        \label{table:state_folded_ac}
        \centering
        \begin{tabular}{cccccccccc}
            \hline
            &Process&$\mathrm{Rohm\,0.18\,\mu m}$&\\
            \hline
            &&&\\
            MOSFET & L[$\mathrm{\mu m}$] & W[$\mathrm{\mu m}$] & finger\\
            \hline \hline
            $\mathrm{M_{p}}$ & 0.72 & 4.27 & 10 \\
            $\mathrm{M_{n}}$ & 0.72 & 4.27 & 20 \\
            $\mathrm{M_{C}}$ & 0.72 & 11.6 & 40 \\
            &&&\\
            $V_{dd}$[V] & $V_{A}$[V] & $V_{B}$[V] & $V_{C}$[V]\\
            \hline\hline
            1.8 & 1.59 & 1.09 & 0.65 \\
            &&&\\
            $I_{bias}$[mA] && $V_{CTRL}$[mV] & \\
            \hline\hline
            4 && 200 & \\
        \end{tabular}
    \end{table}
    \begin{figure}[H]
        \begin{center}
            \includegraphics*[width = 80mm]{figures/vout_ac_theoretical_monochrome.PNG}
            \caption{\eqref{eq:vout}式の周波数特性}
            \label{fig:vout_theoretical}
        \end{center}
    \end{figure}
    \begin{figure}[H]
        \begin{center}
            \includegraphics*[width = 80mm]{figures/vout_ac_simulation_monochrome.PNG}
            \caption{シミュレーションから得られた周波数特性}
            \label{fig:vout_sim}
        \end{center}
    \end{figure}
    どちらも下から$V_{CTRL}=20\,\mathrm{mV},40\,\mathrm{mV},60\,\mathrm{mV}\cdots200\,\mathrm{mV}$のときの周波数特性である。二つを比較すると低域利得には数dBの差があるが遮断周波数はおおよそ近い結果が得られた。\eqref{eq:vout}式では虚部にのみ周波数を含んでいるので、やはり遮断周波数は虚部である寄生容量の値が影響していると考えられる。したがって折り返し型において周波数特性を改善する方法として寄生容量を小さくする、即ちサイズを小さくするといったことが考えられる。


\section{負荷にインダクタを負荷した際の周波数特性}
    前章ではトランジスタサイズを小さくし、寄生容量を減らすことで周波数特性改善につながるということについて述べたが、別の方法としてインダクタを付加するという方法が考えられる。集積化を目標にしているため大きなインダクタを乗せることはできないが、シミュレーションによる改善効果を検討することで現実的なサイズのインダクタを付加することが有効な改善方法になり得るかを検討する。図\refeq{fig:inductor}にインダクタを付加した小信号等価回路を示す。
    \begin{figure}[H]
        \begin{center}
            \includegraphics*[width = 80mm]{figures/ParasiticInductorHalfEquivalent.png}
            \caption{負荷としてインダクタを挿入した小信号等価回路}
            \label{fig:inductor}
        \end{center}
    \end{figure}
    このとき、インダクタは負荷抵抗に直列に挿入したので、\eqref{eq:vout}式の負荷抵抗に$R_{L}+j\omega L$を代入することで利得が求められる。したがって、インダクタを付加した際の利得は
    \begin{multline}
        v_{out,i} \\
        = \frac{4K(R_{L}+j\omega L)g_{mn}\cdot V_{CTRL}\cdot v_{in}}{ \left\{ 1+j2\omega(R_{L}+j\omega L)(C_{SD}+C_{GD}) \right\}\left\{ g_{mp}+j\omega(C_{SD}+C_{SG}) \right\} }    \label{eq:vout_inductor}
    \end{multline}
    と求められた。コントロール電圧を$200\,\mathrm{mV}$に固定し、それ以外の条件は\refeq{table:state_folded_ac}と同様の条件でインダクタを付加した際の\eqref{eq:vout_inductor}式から得られる周波数特性を破線で、シミュレーション波形を実践で図\refeq{fig:inductor}に示す。ただし、どちらにおいても最も下の線が$10\,\mathrm{nH}$であり、上に行くにしたがって$5\,\mathrm{nH}$づつ大きくなる。
    \begin{figure}[H]
        \begin{center}
            \includegraphics*[width = 80mm]{figures/inductor_monochrome.PNG}
            \caption{\eqref{eq:vout_inductor}式の数波数特性}
            \label{fig:inductor}
        \end{center}
    \end{figure}
    理論式とシミュレーションではピーク周波数も異なっていた。\eqref{eq:vout_inductor}式の分母には他にもバックゲート間容量やオーバーラップ容量、ジャンクション容量といった寄生容量がつくことが考えられる。これらが$\omega L$と加算、乗算されることによりピーク周波数もずれてしまったのではないかと考えらる。
    インダクタを挿入した場合、$2\,\mathrm{GHz}$周辺での利得の変化が激しく実際に使用できる周波数はたとえ数$10\,\mathrm{nH}$のインダクタを付加してもそれほど増加しないことが分かった。したがって、現状の折り返し型ギルバートセルにインダクタを付加したところで周波数特性はあまり大きく改善しないことが予想される。


\section{新規トポロジの提案}
    1・2章では折り返し型ギルバートセルの周波数特性改善方法について検討していたが、現状ではサイズを変更することのみが効果的な方法であると思われる。しかし、周波数特性劣化の原因はPMOSにあると考えれらることから、PMOSを使わずにギルバートセルを構成することが最も大きく周波数特性に影響を与えられると考えられる。そこで、図\refeq{fig:NtoN_folded}に今回提案するPMOSを使用しない折り返し型ギルバートセルを示す。
    \begin{figure*}[t]
        \begin{center}
            \includegraphics*[width = 160mm]{figures/NtoNFolded.png}
            \caption{提案する折り返し型ギルバートセル}
            \label{fig:NtoN_folded}
        \end{center}
    \end{figure*}
    提案回路ではカレントミラーと差動対で電流源からの電流を分流することにより、差動対での信号成分を右のカレントミラーに伝え、$\mathrm{M_{A1}~M_{A4}}$のバイアス電流を変化させる。さらにカレントミラーでは左の差動対のソース電位よりもしきい値電圧分右の差動対のソース電位を下げることができる。これによりPMOSを使用せず(周波数特性を従来型と比べ悪化させず)、信号振幅を確保することができるのではないかと予想した。しかしながら、合計で$3I_{bias}$の電流を流す。従来型の消費電流は$I_{bias}$なのでこのままの設計だと3倍の消費電力になる。また、カレントミラーについても信号を伝えるので、このカレントミラーのサイジングも周波数特性に影響を与えることが予想される。
    新しい構成でもカレントミラーを含め完全差動で動作する。したがって解析では半回路を考えれば良い。図\refeq{fig:NtoN_half}に考える半回路を、図\refeq{fig:NtoN_half_equal}にその小信号等価回路を示す。
    \begin{figure}[H]
        \begin{center}
            \includegraphics*[width = 80mm]{figures/NtoNHalfDiff.png}
            \caption{提案回路の差動半回路}
            \label{fig:NtoN_half}
        \end{center}
    \end{figure}
    \begin{figure}[H]
        \begin{center}
            \includegraphics*[width = 80mm]{figures/NtoNHalfDiffEqual.png}
            \caption{差動半回路の小信号等価回路}
            \label{fig:NtoN_half_equal}
        \end{center}        
    \end{figure}
    小信号等価回路において$\Delta g_{m}$は$V_{CTRL}$によるトランスコンダクタンスの変化分であり、前回のゼミ発表より$\Delta g_{m}=2KV_{CTRL}$と表せる。


\section{おわりに}


\end{document}