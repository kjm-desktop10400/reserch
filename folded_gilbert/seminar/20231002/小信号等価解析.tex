
\section{小信号等価解析}
次に寄生容量がついていない場合と各端子間に寄生容量がついた場合の小信号等価解析を行う。解析の際には図\refeq{fgi:base_equiv}に示す小信号等価回路の各端子に寄生容量を付加し、それぞれの場合において計算した。

\begin{figure*}[!t]
    \centering
    \includegraphics*[width=140mm]{figures/FoldedGilbertBaseEquivalentCircuit.png}
    \caption{小信号等価回路}
    \label{fgi:base_equiv}
\end{figure*}

\subsection{寄生容量のついていない場合}
    左側の小信号等価回路について考える。各部に流れる電流は

    \begin{align*}
        i_{ul} = g_{mn}-v_{in}-g_{dn}v_{pS1}\\
        i_{ll} = (g_{mp}+\Delta g_{m})v_{pS1}+g_{dp}(v_{pS1}-v_{out+})\\
        i_{lr} = (g_{mp}-\Delta g_{m})v_{pS1}+g_{dp}(v_{pS1}-v_{out-})
    \end{align*}
    KCLより
    \begin{align}
        i_{ul} &= i_{ll}+i_{lr}   \notag\\
        g_{mn}-v_{in}-g_{dn}v_{pS1} &= 2g_{mp}v_{pS1}+2g_{dp}v_{pS1}    \notag\\
        v_{pS1} &= \frac{g_{mn}}{ 2g_{mp}+g_{dn}+2g_{dp} }v_{in}    \notag\\
        &\approx \frac{g_{mn}}{2g_{mp}}v_{in}   \label{eq:vps1}
    \end{align}
    差動回路なので$v_{out+}=v_{out-},i_{ll}=-i_{rl},i_{lr}=-i_{rl}$となる。これにより
    \begin{align*}
        i_{out+}=i_{ll}+i_{rl} &= i_{ll}-i_{lr}   \\
        &= 2\Delta g_{m}v_{pS1}-2g_{dp}v_{out+}     \\
        \therefore i_{out-} = -2\Delta g_{m}v_{pS1}-2g_{dp}v_{out-}    
    \end{align*}
    したがって、
    \begin{align}
        v_{out} &= v_{out+}-v_{out-}    \notag\\
        &= R_{L}(i_{out+}-i_{out-})     \notag\\
        &= R_{L}\left\{ 4\Delta g_{m}v_{pS1} -2g_{dp}(v_{out+}-v_{out-}) \right\}   \notag\\
        v_{out} &= \frac{4R_{L}}{1+2R_{L}g_{dp}}\Delta g_{m}v_{pS1}     \notag\\
        &\approx \frac{4KR_{L}}{1+2R_{L}g_{dp}}\cdot\frac{g_{mn}}{g_{dp}}\cdot V_{CTRL}v_{in}   \label{eq:vout_base}
    \end{align}

    と求められた。

\subsection{}

