\documentclass[twocolumn]{jsarticle}
\usepackage[dvipdfmx]{graphicx}
\usepackage{kenzemi}
\usepackage{amsmath}
\usepackage{mathtools}
\usepackage{nidanfloat}

\begin{document}

\title{折り返し型ギルバート乗算回路の周波数特性劣化}
\author{小島 光}
\date{2023年10月2日}
\abstract{以前から検討している折り返し型ギルバート乗算回路について、非線形な動作をする乗算回路について小信号等価回路に手を加え解析を行った。またPMOSの各端子間にキャパシタを付加し、各回路について小信号解析を行った。}
\keyword{ギルバート乗算回路,小信号解析,非線形回路}
\maketitle

\section{はじめに}
以前設計した従来型との比較のために設計した折り返し型ギルバート乗算回路(図\eqref{fig:folded_gilbert})ではサイズの大きなPMOSを使用している。したがって、ゲート面積が増大し寄生容量が大きくなることが考えられる。今週はこの寄生容量について、具体的にどの部分が最も周波数特性劣化の原因として大きいのかを検討した。

\begin{figure}[b]
    \begin{center}
        \includegraphics*[width=80mm]{figures/folded_gilbert.png}
        \caption{検討している折り返し型ギルバート乗算回路}
        \label{fig:folded_gilbert}
    \end{center}
\end{figure}

\section{動作点の変動}
図\eqref{fig:folded_gilbert}ではPMOSFETのゲートに直流で$\pm V_{CTRL}$なる電圧を印加する。これはつまり、PMOSFETに流れるバイアス電流が$V_{CTRL}$により変動することを意味している。小信号解析では入力信号が小振幅であるとき、MOSFETのトランスコンダクタンス$g_{m}$が一定であるという近似を前提として小信号等価回路を作成する。しかし、トランスコンダクタンス$g_{m}$は実際には一定でなく、ドレイン電流$I_{D}$と定数係数$K$を用いて

\begin{align}
    g_{m}=2\sqrt{KI_{D}} \label{eq:gm}
\end{align}

と表される。即ち$V_{CTRL}$によりPMOSFETに流れるドレイン電流が変化すると、それに伴いトランスコンダクタンスが変化することになる。このままでは小信号解析を行うことができないので、まずは$V_{CTRL}$とトランスコンダクタンスの関係を明らかにする。\par
まずは片方のPMOS差動対について考えるが、今回のような対応する接点の電圧と電流の符号が反転するような回路において、対応する半分ののみを考えればもう半分については符号を入れ替えればよいので、片側のPMOS差動対のみについて解析すれば事足りる。したがって、図\eqref{fig:folded_gilbert}において左側のPMOS差動対を取り出し、図\eqref{fig:pmos_diff}のような回路について考える。

\begin{figure}[h]
    \begin{center}
        \includegraphics*[width=80mm]{figures/pmos_diff.png}
        \caption{PMOS差動対}
        \label{fig:pmos_diff}
    \end{center}
\end{figure}%

一般にMOSFETのドレイン電流は2乗則の式に従い

\begin{align}
    I_{D}=K(V_{GS}-V_{th})^{2}  \label{eq:square}
\end{align}

である。では、図\eqref{fig:pmos_diff}において$V_{CTRL}=0$のとき、ゲートソース間電圧は$V_{GS}=V_{pS}-V_{G}$なので、トランスコンダクタンスは

\begin{align}
    g_{mp} = \frac{\partial I_{D}}{\partial V_{GS}} &= \frac{\partial}{\partial }V_{GS} K(V_{GS}-V_{th})^{2} \notag\\
    &=2K(V_{GS}-V_{th})    \notag\\ 
    &=2K(V_{pS}-V_{G}-V_{th})   \label{eq:g_mp}
\end{align}

と計算できる。今度は$V_{CTRL}\neq0$の時、$V_{GS}=V_{pS}- \left( V_{G}-V_{CTRL} \right)$なので左右のトランスコンダクタンスをそれぞれ$g_{mpl},g_{mpr}$とすると

\begin{align}
    g_{mpl} = \frac{\partial I_{D}}{\partial V_{GS}} &= \frac{\partial}{\partial }V_{GS} K(V_{GS}-V_{th})^{2} \notag\\
    &=2K(V_{GS}-V_{th})    \notag\\ 
    &=2K(V_{pS}-V_{G}+V_{CTRL}-V_{th})  \notag\\
    &=g_{mp}+2KV_{CTRL}   \label{eq:g_mpl}\\
    g_{mpr} = \frac{\partial I_{D}}{\partial V_{GS}} &= \frac{\partial}{\partial }V_{GS} K(V_{GS}-V_{th})^{2} \notag\\
    &=2K(V_{GS}-V_{th})    \notag\\ 
    &=2K(V_{pS}-V_{G}-V_{CTRL}-V_{th})  \notag\\
    &=g_{mp}-2KV_{CTRL}   \label{eq:g_mpr}
\end{align}

と表すことができた。従って$V_{CTRL}$が差動で印加されたとき、トランスコンダクタンスは$V_{CTRL}$に比例して変化することが分かった。

\section{回路全体の小信号解析}
前章でPMOSについても小信号等価回路で表現することができるということを示すことができた。これにより回路全体についての小信号等価回路を図3に示す。

\begin{figure}[b]
    \begin{center}
        \includegraphics*[width=80mm]{figures/FoldedGilbertBaseEquivalentCircuit.png}
        \caption{回路全体の小信号等価回路}
    \end{center}
\end{figure}

回路全体を小信号等価回路に置き換えることはできたがこれでもまだ複雑である。しかし、本回路中で対応する部分の電流・電圧はそれぞれ異符号かつ等しい絶対値で動作する。従て、対応する部分の片側だけを考えれば事足りる。つまり今回は図4のように$M_{n1},M_{p1},M_{p2}$の小信号等価回路で考えた。

\begin{figure}[h]
    \begin{center}
        \includegraphics*[width=80mm]{figures/FoldedGilbertHalfBaseEquivalentCircuit.png}
        \caption{半回路の小信号等価回路}
    \end{center}
\end{figure}

各部の電流は

\begin{align}
    i_{ul}&=g_{mn}v_{in}-g_{dn}v_{pS1} \label{eq:iul}\\
    i_{ll}&=(g_{mp}+\Delta g_{m})v_{pS1}+g_{dp}(v_{pS1}-v_{out-}) \label{eq:ill}\\
    i_{lr}&=(g_{mp}-\Delta g_{m})v_{pS1}+g_{dp}(v_{pS1}-v_{out+}) \label{eq:ilr}
\end{align}

また、KCL,式\eqref{eq:iul},式\eqref{eq:ill},式\eqref{eq:ilr}より

\begin{align}
    i_{ul}&=i_{ll}+i_{lr} \notag\\
    g_{mn}v_{in}-g_{dn}v_{pS1}&=2g_{mp}v_{pS1}+2g_{dp}v_{pS1} \notag\\
    v_{pS1}&=\frac{g_{mn}}{2g_{mp}}v_{in} \label{eq:vps}
\end{align}

と分かった。ここで差動回路なので$v_{out-}=-v_{out+},i_{ll}=-i_{rr},i_{lr}=-i_{rl}$であることを用いると左側の負荷抵抗$R_{L}$を流れる電流$i_{out-}$は下向きに

\begin{align}
    i_{out-}&=i_{ll}+i_{rl} \notag\\
    &=i_{ll}-i_{lr} \notag\\
    &=2\Delta g_{m}v_{pS1}-2g_{dp}v_{out-}
\end{align}

差動回路なので左側の負荷抵抗を流れる電流$i_{out+}$は

\begin{align}
    i_{out+}&=-i_{out-} \notag\\
    &=-2\Delta g_{m}v_{pS1}+2g_{dp}v_{out-} \notag\\
    &=-2\Delta g_{m}v_{pS1}-2g_{dp}v_{out+}
\end{align}

と表すことができる。$v_{out}\equiv v_{out+}-v_{out-}$とすと

\begin{align}
    v_{out}&=R_{L}i_{out+}-R_{L}i_{out-} \notag\\
    \frac{v_{out}}{R_L}&=i_{out+}-i_{out-} \notag\\
    &=-4g_{mp}v_{pS1}-2g_{dp}v_{out} \notag\\
    v_{out}&=\frac{-4R_{L}}{1+2R_{L}g_{dp}}\Delta g_{m}v_{pS1} \notag\\
    &\approx \frac{-4R_{L}g_{mn}}{1+2R_{L}g_{dp}}v_{in}\Delta g_{m}
\end{align}

ここで$\Delta g_{m}$は$V_{CTRL}$に比例するので、出力電圧は

\end{document}